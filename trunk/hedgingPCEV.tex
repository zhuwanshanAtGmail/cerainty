%%%%%%%%%%%%%%%%%%%%%%%%%%%%%%%%%%%%%%%%%%%%%%%%%%%%%%%%%%%%%%%%%%%%%%%%%%%%
%% Author template for Management Science (mnsc) for articles with e-companion (EC)
%% Mirko Janc, Ph.D., INFORMS, pubtech@informs.org
%% ver. 0.91, March 2007
%%%%%%%%%%%%%%%%%%%%%%%%%%%%%%%%%%%%%%%%%%%%%%%%%%%%%%%%%%%%%%%%%%%%%%%%%%%%
%\documentclass[mnsc]{informs2}              % for a regular run
%\documentclass[mnsc,nonblindrev]{informs2} % for review, not blinded
%\documentclass[mnsc,blindrev]{informs2}    % for review, blinded
%\documentclass[mnsc,copyedit]{informs2}    % spaced for copyediting
\documentclass[mnsc,nonblindrev,copyedit]{informs2_wz} % format for MS submission

% If hyperref is used, dvi-to-ps driver of choice must be declared as
%   an additional option to the \documentstyle. For example
%\documentclass[dvips,mnsc]{informs2}      % if dvips is used
%\documentclass[dvipsone,mnsc]{informs2}   % if dvipsone is used, etc.

% Private macros here (check that there is no clash with the style)

% Natbib setup for author-year style
\usepackage{natbib}
 \bibpunct[, ]{(}{)}{,}{a}{}{,}%
 \def\bibfont{\small}%
 \def\bibsep{\smallskipamount}%
 \def\bibhang{24pt}%
 \def\newblock{\ }%
 \def\BIBand{and}%

%% Setup of theorem styles. Outcomment only one.
%% Preferred default is the first option.
\TheoremsNumberedThrough     % Preferred (Theorem 1, Lemma 1, Theorem 2)
%\TheoremsNumberedByChapter  % (Theorem 1.1, Lema 1.1, Theorem 1.2)
\ECRepeatTheorems

%% Setup of the equation numbering system. Outcomment only one.
%% Preferred default is the first option.
\EquationsNumberedThrough    % Default: (1), (2), ...
%\EquationsNumberedBySection % (1.1), (1.2), ...

% For new submissions, leave this number blank.
% For revisions, input the manuscript number assigned by the on-line
% system along with a suffix ".Rx" where x is the revision number.
% \MANUSCRIPTNO{MS-0001-1922.65}




%%%%%%%%%%%%%%%%%%%%%%%%%% used in original paper %%%%%%%%%%%%%%%%%%%%%%%%
%\documentclass[11pt, a4paper]{article}
%\linespread{1.2} \topmargin -0.25in
%\textheight 8.9in \textwidth 6.5in \oddsidemargin 0in
%\setlength{\voffset}{-0.3in}
%\setcounter{page}{0}


%%%%%%%%%%%%%%%%%%%%%%%%%%%%%%%%%%%%%%%%%%%%%%%%%%%%%%%%%%%%%%%%%%%%%%%%%%%%%%




%\usepackage[pdftex]{graphicx}
\include{epsf}
\usepackage{verbatim}
\usepackage[mathscr]{eucal}
\usepackage{amsfonts}
\usepackage[dvips]{epsfig}
%\usepackage{latexsym}
%\usepackage{amsmath}
%%%%%%%%%%%%%%%%%%%%%%new command%%%%%%%%%%%%%%%%%%%%%%%%%%%
\newcommand{\intinf}{\int_{0}^{\infty}}
\newcommand{\intuk}{\int_{0}^{u_k}}
\newcommand{\dd}{\mathrm{d}}
\newcommand{\E}{\mathrm{E}}
%\newcommand{\argmin}{\mathrm{argmin}} Defined already
\newcommand{\proof}{\noindent{\bf Proof: } }
\newcommand{\qed}{ \hfill $\Box$ }
\newcommand{\OUT}[1]{}
\newtheorem{Def}{Definition}
%\newtheorem{theorem}{Theorem}
\newtheorem{rem}{Remark}\nonumber
%\newtheorem{lemma}{Lemma}
\newtheorem{fact}{Fact}
\newtheorem{pro}{Proposition}
\newtheorem{cor}{Corollary}
\newcommand{\xiv}{\mbox{\boldmath$\xi$}}
\newcommand{\etav}{\mbox{\boldmath$\eta$}}
\newcommand{\lambdav}{\mbox{\boldmath$\lambda$}}
\newcommand{\alphav}{\mbox{\boldmath$\alpha$}}
\newcommand{\rhov}{\mbox{\boldmath$\rho$}}
\newcommand{\rhob}{\mbox{\boldmath$\bar{\rho}$}}
\newcommand{\qv}{\mbox{\boldmath$q$}}
\newcommand{\sv}{\mbox{\boldmath$s$}}
\newcommand{\sigmav}{\mbox{\boldmath$\sigma$}}
\newcommand{\V}{\cal{V}}

%\input{psfig.tex}

%%%%%%%%%%%%%%%%%%%%%%%%% original set, commented out to meet management science  requirment%%%%%%%%%
%\parskip 0.08in
%\title{Optimal Operational {\it versus} Financial Hedging for a Risk-averse Firm}
%\author{Wanshan Zhu \hspace{.2in} Roman Kapuscinski}
%\date{\today}
%%%%%%%%%%%%%%%%%%%%%%%%% original set, commented out to meet management science  requirment%%%%%%%%%


%%%%%%%%%%%%%%%%
\begin{document}
%%%%%%%%%%%%%%%%

% Outcomment only when entries are known. Otherwise leave as is and
%   default values will be used.
%\setcounter{page}{1}
%\VOLUME{00}%
%\NO{0}%
%\MONTH{Xxxxx}% (month or a similar seasonal id)
%\YEAR{0000}% e.g., 2005
%\FIRSTPAGE{000}%
%\LASTPAGE{000}%
%\SHORTYEAR{00}% shortened year (two-digit)
%\ISSUE{0000} %
%\LONGFIRSTPAGE{0001} %
%\DOI{10.1287/xxxx.0000.0000}%

% Author's names for the running heads
% Sample depending on the number of authors;
% \RUNAUTHOR{Jones}
 \RUNAUTHOR{Zhu and Kapuscinski}
% \RUNAUTHOR{Jones, Miller, and Wilson}
% \RUNAUTHOR{Jones et al.} % for four or more authors
% Enter authors following the given pattern:
%\RUNAUTHOR{}

% Title or shortened title suitable for running heads. Sample:
% \RUNTITLE{Bundling Information Goods of Decreasing Value}
% Enter the (shortened) title:
\RUNTITLE{Optimal Operational {\it versus} Financial Hedging}

% Full title. Sample:
% \TITLE{Bundling Information Goods of Decreasing Value}
% Enter the full title:
\TITLE{Optimal Operational {\it versus} Financial Hedging for a Risk-averse Firm}

% Block of authors and their affiliations starts here:
% NOTE: Authors with same affiliation, if the order of authors allows,
%   should be entered in ONE field, separated by a comma.
%   \EMAIL field can be repeated if more than one author
\ARTICLEAUTHORS{%
\AUTHOR{ \today \\Wanshan Zhu}
\AFF{Lee Kong Chian School of Business, Singapore Management University, Singapore 178899, \EMAIL{adamzhu@smu.edu.sg}} %, \URL{}}
\AUTHOR{Roman Kapuscinski}
\AFF{University of Michigan Business School, Ann Arbor, MI 48109, \EMAIL{kapuscin@bus.umich.edu}}
% Enter all authors
} % end of the block




\ABSTRACT{
A Multinational Risk-averse Newsvendor (MRN) produces goods at home (domestically) and sells both overseas and at home, over multiple periods.  The MRN faces risks due to uncertain exchange rate, as well as uncertain demand.  Intuitively exchange rate risk should be managed by a finance department as financial risk, while uncertain demand risk should be managed by an operations department as a part of operational risk.  Traditionally this is the practice of many firms.  In this paper, we consider both of these risks jointly and investigate the effectiveness of two specific alternatives that allow MRN to reduce the total risk.

The first alternative is general financial hedging contracts (including futures, forwards, and swaps).  The second one is operational hedging, which is based on optimally allocating production capacity between domestic and overseas facilities.  We characterize the optimal capacity allocation decisions, financial hedging decisions, and the underlying production and transshipment decisions in a generalized model.  Then, we compare the relative weaknesses and strengths of financial hedging and operational hedging.  Our analysis allows for the case of either exogenous or endogenous prices.  We show *** INCLUDE HERE A FEW MAIN LESSONS *** The general lessons are similar in both cases.
}

% Sample
%\KEYWORDS{deterministic inventory theory; infinite linear programming duality;
%  existence of optimal policies; semi-Markov decision process; cyclic schedule}

% Fill in data. If unknown, outcomment the field
\KEYWORDS{operational hedging, financial hedging, operations management, risk analysis, present certainty equivalent value - PCEV}

%\HISTORY{This paper was first submitted on April 12, 1922 and has been with the authors for 83 years for 65 revisions.}



\maketitle

\section{Introduction \label{sect:intro}}

Over the past two decades firms increasingly take advantage of overseas markets to both produce and sell goods.  Overseas markets may, however, expose firms to the risk of an uncertain and even volatile exchange rates, which amplify the uncertainty in the earnings.  Most firms do care about this uncertainty and attempt to decrease it, in order to reduce the chances of a financial distress or to directly increase the value of the firm.  The company with decreased exposure to financial distress will have a lower cost of capital and a larger credit line and, thus, will provide a larger tax shield for the equity shareholder.  Both the lower cost of capital and the larger tax shield are believed to increase the shareholder's value, as discussed by Brealey and Myers \cite{Brealey2003}.

Two alternative tools are available for the MRN to hedge her exchange rate exposure.  The first is financial hedging, which is often intended to explicitly target exchange rate risks.  The second one is operational hedging, which assumes that total production capacity can be allocated between domestic and overseas facilities so that net foreign cash flow can be reduced, and hence, exchange rate risks are reduced.  Both of these methods are frequently used in practice.

In the past two decades, as the result of breakthroughs in information technology, the variety of financial hedging contracts has grown tremendously and the cost to engage in them has become low.  Consequently, many multinational enterprises have been using some of these contracts, often futures or forward contracts, to reduce their exchange risk exposures.  For example, Cisco Systems, Inc. states in its 2009 annual report \cite{cisco}  ``To reduce variability in operating expenses and service cost of sales caused by non-U.S.-dollar denominated operating expenses and costs, we hedge certain foreign currency forecasted transactions with currency options and forward contracts."  It seems intuitive that the financial hedging allows the MRN to benefit from all the advantages of the profitable overseas market, while not suffering from the consequences of a volatile exchange rate.  

Operational responses experienced a similar growth: an increasing number of global firms place production facilities in multiple countries, capacity is divided between domestic and overseas locations, and transshipment is used whenever needed.  While so-defined operational hedging may incur higher initial cost due to the need to establish overseas production capacity, it may also benefit the MRN in two ways.  First, it allows the MRN to take advantage of the exchange rate movement by producing at a cheaper location.  Second, it reduces the exchange rate risk exposure by using a part of overseas revenue for overseas operations costs.  

Finance literature studies extensively the effects of currency-exchange risk by use of financial hedging, assuming risk-averse decision maker, see for example multiple papers in \cite{Fhedge1, Fhedge2}.
Similarly, the operational hedging has been studied in operations management literature under risk-neutral assumptions, although under various names.  

It is interesting that most of financial literature considers only financial instruments to deal with exchange-rate risk, while most of operations literature considers only operational instruments.  Very few efforts have been made to assess their relative strengths and weaknesses in an integrated framework.  The objective of this paper is to investigate what are the favorable conditions for the MRN to engage in either financial hedging or operational hedging, or both.

%Our basic model is motivated by a risk-averse firm that serves both domestic and overseas demand, where demands and exchange rate are uncertain.  Instead of locating all capacity at home, the firm considers distributing total capacity across domestic and foreign market, which we label as operational hedging.  


In order to analyze the problem, we consider the following general model.  
The MRN produces in two facilities, a domestic facility (DF) and an overseas facility (OF).  The same product serves demands in both markets, domestic market (DM) and overseas market (OM),  see Figure \ref{figure:frameGeneral}.  
\begin{figure}[ht]
    \begin{center}
    \epsfig{file=modelGeneral.eps,width=4.5in,height=3.0in}
    \end{center}
    \caption{General Model}\label{figure:frameGeneral}
\end{figure}

The MRN has a finite capacity.  In the beginning of the horizon, the MRN allocates capacity between DF and OF and then operates the system across multiple periods.  Demands in each period are uncertain and exchange rate changes from period to period according to Markov-modulated process. DM's demand can be satisfied by DF and, if needed, by OF's production using transshipment.  The same applies to overseas location.  In each period, MRN decides how much to produce in each facility, how much to transship and what financial contract to sign. 

Both types of hedging that we study are special cases of this model:  {\em operational hedging} means that we allocate the total capacity between both locations and consequently produce both in DF and OF, but use no financial hedging; {\em financial hedging} means that all capacity is allocated to DF (consequently, we produce in DF only), while we are allowed to use any financial contracts in any period.

A critical element of our model is MRN's risk-aversion. Even though it is argued that a firm should be risk neutral to non-systematic risks because the shareholders may be able to diversify away those risks, Smith \cite{Smith2004} in his review of risk attitudes points out that many firms, including large public companies, are risk-averse.  Some practitioners of decision analysis (see Bickel et al.  \cite{Bickel2002}) suggest deriving a firm-wide risk-averse utility function.  
Second critical element of the model is finite capacity.  
Due to finite capacity, some demand may be not satisfied and, therefore, lost. Transshipment and production costs are linear.
The basic model assumes that price is exogenous.  As an extension, we consider price-sensitive demand, where MRN can adjust price from period to period.


Having introduced the problem, the rest of the paper seeks to analyze it.  


**** Contributions *****

- answer practically relevant question

- not analyzed directly in literature

- show structure, explain the nature of financial hedging and provide intuition

- provide conditions when simple forward is effective

- Characterize performance of PCEV in this specific context and describe properties of PCEV with respect to parameters of the problem

- characterize when each of the hedging is most beneficial ....

- describe the effect of time horizon.

EDIT PARAGRAPH BELOW WHEN DONE:
As a background, Section \ref{sect:liter} contains discussion of literature. In Section \ref{sect:pcev} we discuss the properties of PCEV. In Section \ref{sect:basicModel} we present structural properties of optimal solution for the Basic Model; while in Section \ref{sect:priceModel}, we do so for the Price Model.  In Section \ref{sect:numer}, we analyze the results of numerical analysis and discuss the relative strengths and weaknesses of the financial and operational hedging in detail.  Finally, in Section \ref{sect:conclusion}, we present our conclusions.





\section{Literature Review \label{sect:liter}}
\OUT{
1. RISK AVERSION - briefly and say that the next section deals in more detail.

2. EXCHANGE RATE

3. TRANSPORTATION

4. FINANCIAL AND OPERATIONAL HEDGING
	EMPIRICAL
	THEORETICAL


Multiple streams of literature are related to our problem.  The first stream introduces various ways of modeling risk-averse attitudes across extended horizon.  The second stream describes analysis of problems that involve exchange-rate uncertainty.  The third stream includes analysis of multi-location problems.  In the final fourth category, we discuss relevant empirical and theoretical literature related to comparison of financial and operational hedging.






\medskip
}
Multiple streams of research are relevant to our topic.  We list them first before describing them in more detail. The first stream deals with multi-location production and its related issues of transshipment and capacity allocation. The second studies the exchange rate uncerainty and its impact on operations under risk-neutral assumption. The third investigates financial and operational hedging empirically and theoretically. The fourth introduces risk-averse models and their strengths and weaknesses. 

*** I EDITED SOME PIECES BELOW, BUT THE FLOW OF LITERATURE REVIEW IS NOT SMOOTH SO FAR ***

%- Currency exchange-rate risk: the use of financial and operational hedging.  

%- Currency exchange rate uncertainty -- considered in both operations and finance literature.  %Most of operations literature assumes risk neutrality, while most of finance literature assumes risk aversion, but does not consider demand uncertainty and operational decisions.



%- The comparisons between financial and operational hedging. % have been considered in very few papers.  We describe them in detail and compare to our model.  

\medskip
\noindent \underline{Multi-location production}

Since we allow production and sales to occur across countries, multi-location production and distribution literature is relevant.  Examples from this broad literature include Krishnan and Rao \cite{Krishnan1965}, who study a single-period two-location problem, as well as extensions to N-locations, or  Tagaras \cite{Tagaras1992}, who study the pooling effects of transshipment on service levels. Recently, Hu {\it et al.} (2007, 2008) study a general transshipment problem in centralized and decentralized settings when capacity may be uncertain. Huang and Sosic (2010) study the game among retailers that order independently before demand realizes, and then transship among them after demand realizes. Similar to these papers, we also consider the product trasshipment across multi-locations.  While these papers focus on the transshipment problem, for us it is a tactical subproblem  -- we concentrate on the impact of currency exchange rate uncertainty and potential hedging policies.  Transshipment is an element of our operational hedging.


\medskip
\noindent \underline{Currency exchange-rate uncertainty} 

A number of operations management papers address production and inventory decisions in the presence of exchange rate uncertainty.  Dasu and Li \cite{Dasu} and Li {\it et al.} \cite{Li2001} focus on the opitmal production policies when the exchange rate uncertainty affects the production cost.  Dasu and Li allows for switchover cost and shows that it is optimal to switch the production locaiton from one to the other if the exchange rate is above one threshold, and to switch back if it is below another thershold. Li {\it et al} assumes no swithover cost and derives the optimal production as a function of initial inventory and exchange rate. Aytekin and Birge \cite{Aytekin2004} study a single capacity-allocation decision, followed by a series of production decisions, when exchange rate is uncertain, but demand is certain.  They show that a band control policy is optimal for production. Kazaz et al. (2005) study the production decisions when a firm has production facilities in two foreign countries and found that the company may underserve the demand even when it has enough capacity because the exchange rate and transshipment cost make serving demand not profitable. Dong et al (2010) study a firm's choice among facility networks of home only, foreign only or both when it faces the exchange rate and demand uncertainty, and show that the expected marginal profit determines the choices.  While all these papers consider the exchange rate uncertainty, they differ from our paper in that they do not consider using the financial instruments to hedge the risks, thus are not able to study the financial hedging and operational hedging in a single framework for a risk-averse decision maker.We next discuss the papers that study both hedges. 



%  HE TOOK IT OUT OF HIS WB PAGE.  I GUESS HE DOES NOT INTEND TO PUBLISH IT ANYMORE.  IT WAS ABOUT TWO LOCATIONS LOCATED IN TWO COUNTRIES.  {\bf Allen Scheller Wolf, note I can not find his paper}


\noindent \underline{Financial hedging and operational hedging}

A recent review by Bandaly et al. \cite{Bandaly_supply_2010} reports that the term operational hedging is used for activities such as geographic dispersion, switching production, capacity allocation postponement. These activities are often characterized by ``mitigating risks by counterbalancing actions,'' see \cite{vanmieghem2003}.  Bandaly et al note that the majority of papers label operational hedging as setting multi-location production (geographic dispersion) instead of single0location production.  %The dominant use of operational hedging fits the context of currency rate uncertainty and is the one we consider in our paper.
There is no ambiguity about financial hedging, which includes general financial hedges, currency derivatives, currency forward, currency options, exotic derivatives, and foreign debt.  Our paper considers allocation of capacity across countries as operational hedging and currency derivatives as financial hedging. There is a considerable empirical literature that studies operational versus financial hedging and it remains very active.


Empirical papers define operational hedging as international and geographical dispersion.  They acknowledge that companies use both financial and operational hedging, and study which of them is more beneficial.  They are not fully consistent in evaluating the benefits.   
Pantzalis et al (2001), find support of hypothesis that the exposure of cash flows to currency risk is effectively managed by operational hedging.
Allayannis et al (2001) find that operational hedging is not an effective substitute for financial hedging and that operational hedging increases firms' values only if used in conjunction with financial hedging.  Both of these papers consider consider data from US companies. 
Kim (2006), Aabo and Simkins (2005), and Faseruk and Mishra (2008) based on US, Danish, and Canadian data, respectively, find that the firms that use (or have a possibility of using) operational hedging are less likely to use financial hedging. Carter et al (2001) suggests that firms that use both hedges face effectively smaller exchange-rate exposure.  A superb summary of empirical literature can be found in Bandaly (2010).  Our paper complements the empirical literature.  Its intent is to explore the role of each of the types of hedging and to evaluate their individual and joint benefits. While these emprical papers come exclusively from financial literature, the theoretical papers that attemptto compare financial and operational hedging appear both in finance and operations management literature. 

%FINANCE:
In theoretical finance literature dealing with currency exchange risk, a few papers consider both demand and exchange rate risks.  
Among those, Logue (1995) suggest that financial instruments cannot effectively hedge operational exposure. 
Chowdry and Howe (1999) in a one-period model use mean-variance as objective to study financial and operational hedging.  The main conclusion is that operational hedging is needed only in the presence of both demand and exchange rate uncertainty.
Hommel (2003) extends the above model by including convex (quadratic) cost for foreign capacity and concludes that the operational hedging is only useful for sufficiently high exchange rate variability. 

LATER (?) : Our paper is different from these papers in several aspects.  First, our primary question of when financial and when operational hedging is most appropriate, is not directly answered (apart of the special case, when of no demand or no exchange-rate uncertainty).\footnote{Also, some of their assumptions are more restrictive than ours leading to not exactly the same conclusions.  First, they do not allow capacity shortage and do not consider the cost of transshipment.  Second, due to assuming single period, decisions have only short-term effect creating potential bias of ignoring dependencies between financial and operational decisions(due, for example, to the fact that in some future periods expected cost of domestic and overseas production might not be equal).}  Second, we consider the dynamic hedging in multi-periods, and our objective function is different.  Consequently, we are able to derive properties of financial hedging not available with mean-variance objective.  Their main finding, that operational hedging is necessary only when {\em both} exchange rate and demand are uncertain, is not true in general in our multi-period model.  We are also able to evaluate how a broader range of factors, such as transshipment costs, possibility of price elasticity, influence the relative benefit of financial and operational hedging.

In operatons management papers, a few substreams study appropriate financial constract without focusing on the relationship between financial and operational hedging.  For example, Gaur and Seshadri \cite{Guar2004} consider a single-period one-facility model with demand and financial asset values being correlated.  Their purpose is to construct the optimal financial hedging contract.  Agrawal and Seshadri \cite{Agrawal2000} investigate joint price and production decisions in similar settings, when demand and financial asset values have correlation and find corresponding optimal hedging contract. Caldentey and Haugh \cite{Caldentey}, who study financial hedging in an incomplete market, where one financial asset and one operational value follow continuous-time correlated diffusion processes and a mean-variance objective function is used.  Chod et al (2010) bears some similarity to our work, as they consider both financial and operational hedging.  Their main question is whether financial hedging and operational flexibility are complements or substitute. In their model, however, the financial instrument affects demands, but not the cost and sales price (*** THEY DO NOT HAVE ANY SELLING PRICE OR COST IN FOREIGN CURRNCY???****). Furthermore, all these paper consider only single produciton facility and do not expressly address the exchange rate uncertainty.


In papers expressly addressing the exchange rate uncertainty, Huchzermeier and Cohen \cite{Huchzermeier} studied the finanical hedigng and operational hedging and show that the operational hedging exploits the volatility to increase the net present value, while financial hedging eliminates the volatility. Their objective is to evaluate the option value of production network to a risk-neutral decision maker who uses only forward contract in financial hedging.  Ding {\it et al.} (2007) study a risk-aversion decision maker's joint decisions of the financial hedging and the optimal postponement in a single-period model and use mean-variance as objective function. Their model allows decision maker to uses a portfolio of call and put options in the financial hedging. However, this portfolio is still less general than our model.  Furthermore, their work {concentrates} on the effect of postponement of production decisions and the optimal option contracts and does not compare financial hedging to operational hedging.
Wang and Li (2010b) extend Ding et al (2007) to consider financial and operational hedging for more than two countries using a mean-variance objective function in a static model. Their financial hedging contract is similar to ours, and thus, more general than put call options in Ding et al. (2007). Then they study a particular configuration where a US company considers setting capacities in China and Vietnam and using financial hedging. They show numerically that financial hedging and operational hedging are substitutes. Our research complements their work by studying a dynamic model. To the best of our knowledge, we are the first to consider the dynamics in hedging.  While they uses mean-variance objective function to model the decision maker's risk-aversion, this method is not the only way for static models and its direct extension to dynamic models has some theoretical difficulties. 

\OUT{
In our model, while we do not allow inventory to be carried across periods, the consecutive periods are still linked due to the dynamics of the exchange rate.  A few papers consider multiple periods, but they consider only one facility (and do not study operational hedging) or do not consider any form of risk aversion.  A paper that considers both financial and operational issues is The focus, however, is not on comparison of two types of hedging, but instead on deriving financial hedging policy.  %We are not aware of any other papers comparing financial {\it versus} operational hedging.





	- related to currency exchangE


\noindent \underline{Within modeling papers,} most of modeling papers in operations literature assume risk neutrality, while most of finance literature assumes risk aversion, but does not consider demand uncertainty and operational decisions.




Despite different objective, 


\medskip


Our paper is different from those cited above in that we allow a decision maker to be risk-averse and that financial hedging contracts are explicitly built into our multi-period operational model.  Our aim is to first understand the structure of optimal policies, and then to compare the relative strength of financial and operational hedging for various environments.  Within our model, based on optimal production decisions, we are able to derive pleasing properties of financial hedging -- it equalizes the consumption cash across periods and across exchange rate realizations. In the numerical study that follows, we compare the benefits due to operational hedging and due to financial hedging, show that in general operational hedging offers significantly higher benefits, and characterize the scenarios when it happens.  In order to measure systematically the effects of risk aversion, in the numerical study we use a family of exponential utility functions.
}

\medskip




\noindent \underline{Risk-aversion Models}

Risk aversion is central to a few sub-streams of economics and finance literature. Recently it has been increasingly studied in operations literature.
\OUT{
 %, although predominantly in single-period models.%  Economics and finance literature has developed criteria to evaluate the appropriateness of various risk-aversion frameworks. 
}
 In this paper we study the decisions of a risk-averse firm facing two uncertainties, stochastic demand and exchange rate over time.  
Therefore, we start by discussing how the risk and time preference are modeled.

Two major approaches to modeling the risk attitude are: (a) Value-at-Risk (VaR), and (b) various forms of utility functions.  VaR is defined as the maximum loss of value that a firm can incur for a given confidence level and a given time interval.  Many financial firms use Value-at-Risk to manage day-to-day operations.  Manganelli and Engle \cite{Manganelli2001} provide a discussion of the VaR models and their uses in practice. Tapiero \cite{tapiero_value_2005} applies the VaR to a single-stage inventory control problem and finds that the target cost level can be interpreted as VaR.  Recently Devalkar et al. \cite{devalkar_integrated_2010} applies VaR to study the optimal decisions of a firm that procures, trade, and process commodities. %Boyabatli and Toktay (2010) further extend this model to consider the choice of flexible and dedicated technologies when the firm faces financial budget constraint.


\medskip

Most theoretical models of risk-averse behavior employ a utility function to study consumption.  Levy \cite{Levy92} surveys the use of utility functions to define the risk.  This approach is based on translating a cash flow $x$ in a given period into utility function $U(x)$, which is concave and increasing.  Levy points out that, ranking risk by non-decreasing concave utility function is consistent with the natural definition of risk premium given by Arrow \cite{Arrow1951} and Pratt \cite{Pratt1964} -- the total utility serves as an index for risk.  Gerber and Pafumi \cite{Gerber1998} review the most often used utility functions: (i) linear, (ii) exponential, (iii) negative quadratic, and (vi) logarithmic.

Obviously, linear utility function means risk neutrality, and it is often implicitly assumed in models that do \emph{not} deal with risk.  Negative quadratic utility is often used in finance literature, see survey by Steinbach \cite{Steinbach2001}.  The popularity of negative quadratic function is partly driven by technical reasons as it considers only the mean and variance of the cash flow. 
Since the quadratic utility function has its obvious shortcoming, penalizing superior performances,  general concave utility functions for risk-averse modeling have been widely used in economics, see survey by Fishburn \cite{Fishburn1989}. In operations literature,  Eeckhoudt {\it et al.} \cite{Eeckhoudt1995} employ a concave increasing utility function to study a typical single-period  news-vendor problem.  They show that a high risk-aversion newsvendor orders less inventory than a low risk-aversion one. Their model is extended to multiple resource capacity investment by Van Mieghem (2007) and to multi-period dynamic inventory control with set-up cost by Chen {\it et al.} (2007).

\OUT{
as well as in operations management, see Echkout and {\it et al.} \cite{Eeckhoudt1995}  and Van Mieghem (2007).  Many of the papers use the exponential utility function in order to provide closed-form solutions for their model, see Gerber and Pafumi \cite{Gerber1998}.
Other papers concentrate specifically on studying risk averseness without financial hedgings.  
}

While the approaches to one-period risk modeling are relatively well agreed on, it is less clear in multi-period model how to capture the time preference combined with risk aversion.  One of the fundamental issues here is the intertemporal consistence, {\it i.e.}, given risk preferences defined over multiple time periods, whether the optimal policy formulated for future periods will be carried out as planned.  In an intertemporally inconsistant model, the decision maker does not implement the optimal policy planned for the future. Since this inconsistence is not deriable, researchers in economic literature typically use discounted utility model that is intertemporally consistent and was first proposed by Samuelson \cite{Samuelson1937}. In this model, total utility is the discounted sum of utilities of cash flows across all time periods. Frederik {\it et al} (2002) review the extensive application of this model and its variatons in economic literature. However, Sobel (2006) show that in a stochastic environment using this model for risk-aversion is not compitable with the decomposition axiom and its converse. A compatible model is the utility of the sum of discounted cash flows, which Bouakiz and Sobel (1992) use to study dynamic inventory control problem. To model risk-aversion, they use a particular utility function, i.e., exponential, and thus achives the intertemporal consistence.  Novertheless, the applicaion of this model is not as extensive as the discount utility model bacause using utility functions other than exponential in general leads to intertemporal inconsistence.

Another fundamental issue in modeling multi-period risk-aversion is the "temporal risk preference" that represents a decision maker's preference in timing of uncertainty resolution. In general an earlier resolution of the uncertainty is prefered becaust it gives the decision maker more information to formulate better plans for the future. Unfortunately, both models of discount utility and utility of discounted cash flows do not capture this temporal risk preference. 

\OUT{
   Later, Samuelson \cite{Samuelson1952} and Pollack \cite{Pollak} argue that, when risk preference is assumed independent of time, the additive utility across time does, in fact,  give consistent choices.  The discounted utility model has been used for several decades, with several generalizations.  The approach was generalized in works of Prakash \cite{Prakash1976} and Fishburn \cite{Fishburn1989}.  They show that under proper assumptions, the consistent choices are possible.  OMIT THE REST?: Prakash \cite{Prakash1976} shows that the consistence of preference across time can be completely determined if (1) The preference across time is specified by a continuous monotone function; and (2) the preference, at a fixed instant of time, is specified.  Fishburn {\it et al.} \cite{Fishburn1982} conclude that, if utility function is monotonic and continuous, then the resultant choice is consistent.  
}
\OUT{
Assuming stationary condition (that is, if $x$ at time $t$ is preferred over $y$ at time $t + k$, then $x$ at time $s$ is preferred over $y$ at time $s +k$), they show that the utility can be expressed as $U(x,t) = \alpha^t f(x)$, where $1>\alpha>0$ and $f$ is an increasing function.  They also present other forms of utility functions under conditions weaker than the stationarity.  
}
\OUT{
Obviously the discounted utility models satisfy the stationarity condition.  Frederick {\it et al.} \cite{Frederick2002} survey the popularity of theoretical economic analysis based on the discounted utility model, as well as some of its anomalies, pointed out by empirical examples.  They discuss a few minor modifications of the discounted utility model.  The distinctions they draw are that (1) the discount rate as a function of time may be decreasing, and (2) the utility across time may  be interdependent, {\it e.g.}, increasing sequence of payoffs is preferred over decreasing one.

Even though the utility approach is mostly used to study the consumption decisions in economics and finance literature, it is also used in a few papers to evaluate a company's value and to study the operations decisions. 

For example, Bouakiz and Sobel \cite{Bouakiz1992} use the expected utility of the total net present value; while this approach does not meet the stationary and separability condition, it meets the continuous and monotonic conditions.
Sobel \cite{Sobel2007} discusses benefits and drawbacks of discounted sum utilities vs. utility of net present value. Utility of net present value has not been widely accepted and drawbacks are reported.  Using this approach to evaluate a company's value by considering only the operations decisions fails to account for the timing of uncertainty resolution, see Matheson and Howard (1968), Mossion (1969) and also illustrated by an example in Smith (1998, page 1690), as the ``temporal risk problem.'' Also, in the case of using expected utility of the total net present value, the optimal operations decisions are not time consistent.
(WANSHAN - DOES THE LAST SENTENCE REPEAT THE SAME MESSAGE AS ABOVE?)
}  


Two modified utility-based approaches are developed to address the problems related to temporal risk preference: recursive utility function and present certainty equivalent value (PCEV). Kreps and Porteus (1978) first proposed recursive utility procedure to account for the timing of uncertainty resolution. Epstein and Zin (1989) generalized their models to allow independent risk-aversion specifications for a given period and across periods. The Epstein-Zin framework has become the most popular model for dynamic asset pricing applications in economic and finance literature; Many textbooks, including Singleton (2006) and Duffie (2001), discussed these applications. Their framework is specified by three elements: a recursive utility function, an aggregator function and a certainty equivalent operator.    The freedom in choices for these elements makes the framwork very powerful, yet too general. It turns out that PCEV corresponds to a particular choice for the elements where the recursive utility and the aggregator functions are additive exponential and the certainty equivalent operator is the PCEV operator. This particular choice  makes the PCEV conducive to efficient evaluation. 
\OUT{
WHAT ARE WE SAYING HERE?  IS IT A REASON TO USE PCEV, OR SIMPLY SOBEL's METHOD IS BAD?  
While the recursive utility theorem by Epstein and Zin (1989) solves this problem of temporal risk problem, it is not in general consistent with the expected utility preference for consumption. 


Present certainty equivalent value (PCEV) approach captures the decisions maker's risk preference that is sensitive to both uncertainty and the time at which the uncertainty is resolved.
It directly models the decision maker's decisions in both consumption and operations (IS HAT RIGHT???? CANT WE USE PCEV OUTSIDE OF OPERATIONS???).


%This recursive PCEV can be viewed as a special case (more amenable to analyze) of the recursive utility in Epstein and Zin's framework. 


In our paper, we follow the most popular and most agreed-for discounted utility model and use PCEV to study the firm's decisions, as it captures the decisions maker's risk preference that is sensitive to both uncertainty and the time at which the uncertainty is resolved. (REPETITION WITH THE BEGINNING OF THE PREVIOUS PARAGRAPH)

%\subsection{ Present Certainty Equivalent Value \label{sect:pcev}}
}


\section{Present Certainty Equivalent Value}

PCEV is a lump-sum amount received with certainty today that generates the same consumption utility as a stream of uncertain operatinal cashflows. Since the decision maker can use financial instruments to smooth the consumption, Dreze and Modigliani (1972) developed the PCEV concept in studying the optimal consumption decision for a two-period uncertain income. Smith (1998) extended their model to multi-period and assuming exponential utility function was able to derive a recursive formula for PCEV. PCEV explicitly models the investments in these instruments and uses utility functions to represent consumption preference of the decision maker. The functions we use are exponential utility functions, following Smith (1998). We consider the evaluation of PCEV under the investment of either risk free bond or exchange rate derivatives.


\subsection{PCEV with Risk-free Bond}

We let $n$ be the index of the current periods and count the periods backwards, with the last period being 0. Let $f_i$ and $\beta_i$ be the cash generated in period $i$, respectively, from operations and from past investment, and let $r_f$ be the bond's risk-free interest rate. The cash consumption in period $i$ is then  $f_i + \beta_{i} -\beta_{i-1}/(1+r_f)$, where the last term is the price paid on the bond investment that generates cash $\beta_{i-1}$ in period $i-1$. Since consumption ends in period $0$, $\beta_{-1} =0$. Thus, for given initial wealth $\beta_{n}$,  the maximum consumption utility of the stream of operational cash flow $f_i$s is
\begin{eqnarray} 
     \label{eqn:bondInv-rho}
\mathcal{U}_n(f_n, \cdots, f_0|\beta_{n}) = \max_{\beta_{i} \forall 0 \leq i < n} \{\E [ \sum_{i=0}^n \big{(}-w_i \exp (-\frac{f_i + \beta_{i} -\beta_{i-1}/(1+r_f)}{\rho_i})\big{)}]\},
\end{eqnarray}
where $w_i$ is utility weight and $\rho_i$ is risk tolerance, which capture the decision maker's time and risk preferences for consumption respectively.  The expectation is taking on all the future uncertainties for given realized foreign currency value in period $n$.

To capture the decision maker's indifference between PCEV and the operational cash flows, the maximum utility generated by PCEV is the same as the maximum utility generated by the operational cash flow, thus we have

\begin{definition}
    \label{def:PCEV}
The present certainty equivalent value at period $n$ of an uncertain operational cash flow $f_n, \cdots, f_0$ is defined as $\V_n$ such that
\begin{eqnarray} 
     \label{eqn:PCEV}
\mathcal{U}_n(\V_n(f_n, \cdots, f_0|\beta_{n}), 0, \cdots, 0|\beta_{n}) = \mathcal{U}_n(f_n, \cdots, f_0|\beta_{n}).
\end{eqnarray}
\end{definition}





\subsection{PCEV with Financial Hedging}

The MRN can improve the PCEV by financial hedging, which is to buy foreign currency derivatives (a future payoff as a function of the future realized value of the foreign currency). The MRN pays the market price of the derivatives, which is valued under risk neutral measure.

Let $S_i$ be the value of foreign currency and $s_i$ be its realization, and let $\beta_{i-1}(S_{i-1})$ be investment payoff in period $i-1$ of a derivative that the MRN buys in period $i$, then its price is $\E[\beta_{i-1}(S_{i-1})|S_i]/(1+r_f)$. As a result of financial hedging, the consumption cash flow in period $i$ is $f_i + \beta_{i}(s_i) - \E[\beta_{i-1} (S_{i-1})|S_i]/(1+r_f)$. Let $\beta_{n}(s_n)$ be the initial cash  and let $\beta_{-1}(\cdot)=0$ due to end of consumption in period $0$, then the maximum utility of a uncertain operational cash flow becomes
\begin{eqnarray} 
    \label{eqn:maxU-FH}
\mathcal{U}_n(f_{n},\cdots,f_0|\beta _{n}(s_n) )  =\max_{\beta_{i}(S_{i}), \forall 0\leq i< n} \{\E [\sum_{i=0}^n \big{(}-w_{i}\exp(-\frac{f_{i} + \beta_{i}(S_i)-\frac{\E[\beta_{i-1}(S_{i-1})|S_i]}{1+r_f}}{\rho_i})\big{)}  |S_n=s_n]\}.
\end{eqnarray}

The corresponding PCEV is then defined in (\ref{eqn:PCEV}) except the bond investment is replaced by exchange rate derivative investment. This definition implies that the $\V_n(\cdot)$ is a function of $s_n$. 





\section{Basic Model \label{sect:basicModel}}

In this section we describe the assumptions and the structural properties of the model that allow for both financial and operational hedging.  In every period firm faces demands in both markets.  Demands are independent from period to period, but not necessarily independent across locations.  The exchange rate follows a Markov-modulated process with a finite number of possible realizations.  After demand and exchange rate are realized, the firm decides production quantities as well as quantities to transship.  No inventory can be held from period to period.  The sales price is assumed to be exogenous.  This assumption is especially appropriate in competitive market, where the MRN may be forced to be a price taker (Section \ref{sect:priceModel} allows MRN to set the price). 

We assume that the objective of the MRN is to maximize her Present Certainty Equivalent Value (PCEV) for all periods of the planning horizon, and that she has finite total capacity. In each period PCEV is calculuated after the realization of exchange rate, but before demand realization.  To maximize PCEV, she has three decisions to make, initial decision regarding allocation of capacity (i.e., operational hedging), and then in every period decisions regarding financial hedging and production/transshipment.

The first, operational hedging, decision is how to allocate the total capacity between DF and OF at the beginning of horizon.  Once this decision is made, DF's and OF's capacities do not change in any later periods. The production/transshipment decision determines how much to produce at each location and how much to transship from a location to the other market. The financial hedging decision is what financial contract, if any, to enter into. The events in each period are as follows. (1) current-period exchange rate is revealed, (2) current-period demand becomes known. (3) production and transshipment decisions are made. (4) operational cash is collected. (5) risk-free bond or trade exchange rate derivatives (Financial hedging) are decided and cash from previous investments is collected and consumed.

The following additional notation is used.  The first six symbols are illustrated in Figure \ref{figure:frameGeneral}.
\begin{itemize}
    \item $d$ and $o$ superscripts of domestic and overseas markets or facilities, respectively,
    \item $\xi^j$ market $j$ demand, ${\xiv} = \{\xi^d, \xi^o \} $,
    \item $k^j$ capacity of facility $j$,   $\bold k = \{k^d,k^o\}$,
    \item $z^{j}$ quantity produced in facility $j$ for market $j$, $\bold z = \{z^{d},z^o\}$,
    \item $x^{j}$ quantity produced in the other facility and transshipped to market $j$, $\bold x = \{x^d,x^o\}$,
    \item $y^{j}= z^j + x^j$ total production quantity available to meet market $j$ demand, $\bold y = \{y^d,y^o\}$,
%\newpage
 %   \item $i$:  subscript index of the period number, counting backwards,


%PUT BACK HERE exchange rate TO cumulative hedging contracts

    \item $K$: total capacity, $K >0$,
    \item $p$: unit price, $p > 0$,
    \item $c$: unit production cost, $c \in (0, p)$,
    \item $t$: transportation cost, $t\geq 0$,
\end{itemize}
The exchange rate $S_i$ is assumed to be independent of demand $\xiv$. The sale price $p$ is assumed to be exogenous in the Basic Model, which is relaxed in the Price Model, see Section \ref{sect:priceModel}. Finally, we assume that no inventory is carried over across periods.  Thus, the only interaction across periods is through the exchange rates and capacity. The MRN's objective is to maximize the PCEV of all future operational cash flows.

\begin{eqnarray}
     \label{eqn:capacityAlloc}
\max_{k^d + k^o \leq K, k^d \geq 0, k^o \geq 0} \V_n(f_n(s_n,\bold{k},\bold{\xi}_n), \cdots, f_i(S_i,\bold{k},\bold{\xi}_i), \cdots, f_0(S_0,\bold{k},\bold{\xi}_0)),
\end{eqnarray}
where one-period maximum operational cash is
\begin{eqnarray}
f_i(s_i,\bold k, \xiv_i) &=& \max_{(\bold{z,x}) \in A (\bold {k, \xiv_i})} \{(p-c)z^{d}+ (p-cs_i-t)x^{d} + s_i(p-c)z^{o}+(s_i p-c-t)x^{o} \},\label{eqn:profit} \\
 A(\bold{k,\xiv_i}) &=& \{ (\bold {z,x}): \; z^{j} + x^{l\neq j} \leq k^j \; \forall j\in \{d,o\}\; \forall l\in \{d,o\}, \; \bold z+ \bold x \leq \xiv_i,\; \bold z \geq 0, \; \bold x \geq 0\}. \label{eqn:prodCt}
\end{eqnarray}

The objective function in (\ref{eqn:profit}) is revenue minus costs for given production decisions ($\bold{z,x}$) in period $i$.  The constraints on ($\bold{z,x}$) in (\ref{eqn:prodCt}) force the production at each facility not to exceed its capacity, and sales at each market not to exceed its demand. The resulted maximum operational cash is a function of the exchange rate and demand only of period $i$ because the production and transshipment decisions of a period are to maximize the operational cash of that period, they have no impact on other period's operational cash.





\section{Analysis}





\subsection{Properties of One-period Profit Function}

To analyze the problem, we need some initial properties of the profit function $f$ in (\ref{eqn:profit}).  These are building blocks for studying later the properties of capacity decisions.  The period index $i$ is suppressed for exchange rate, {\it i.e.}, $s= s_i$, unless specified otherwise.

{\lemma $f$ is non-decreasing and concave in $(\bold k, \xiv$).
\label{lemma:conProfit}}

\proof See Appendix.  \qed

Lemma \ref{lemma:subProfit} below formally states that the groups of capacity and demand decisions are complementary, but within each group (demands or capacities) they are substitutes.


{\lemma $f$ is submodular in $(\bold k, -\xiv)$.
\label{lemma:subProfit}}

\proof See Appendix.  \qed

\OUT{
\proof For simplicity of the presentation, let $\alpha_{jl}$ be the coefficient of decision variable of objective function in (\ref{eqn:profit}), {\it i.e.}, $\alpha_{dd} = p-c, \; \alpha_{od} = p-cs-t,\; \alpha_{do} = sp -c -t, \; \alpha_{oo} = s(p-c)$.  To prove submodularity, we first write the dual of the production problem:
\begin{eqnarray*eqnarray*}
\begin{array}{lll}
    f(s,\bold k, \xiv) &=& \min_{(\etav, \lambdav) \in B (\alphav)} G (\etav, \lambdav, \bold k ,\xiv)  \\
    G(\etav,\lambdav,\bold k, \xiv) &=& \etav \bold k + \lambdav \xiv \\
    B (\alphav) &=& \{\eta_j+\lambda_l  \leq \alpha_{jl} \; \forall j \in \{d,o\}, \; \forall l \in \{d,o\},\;  \etav \geq 0,\; \lambdav \geq 0 \}
\end{array}
\end{eqnarray*eqnarray*}
To see the relationship between $\bold k$ and $-\xiv$, we substitute $\bar{\xiv} = - \xiv$ and $\bar{\etav} = - \etav$.  Then, the dual problem becomes:
\begin{eqnarray*eqnarray*}
\begin{array}{lll}
    f(s,\bold k, \xiv) &=& \min_{(\bar{\etav}, \lambdav) \in B_1 (\alphav)} G (\bar{\etav}, \lambdav, \bold k ,\bar{\xiv})  \\
    G(\bar{\etav},\lambdav,\bold k, \bar{\xiv}) &=& -\bar{\etav} \bold k - \lambdav \bar{\xiv} \\
     B_1(\alphav) &=& \{-\bar{\eta}_j+\lambda_l  \leq \alpha_{jl} \; \forall j \in \{d,o\}, \; \forall l \in \{d,o\},\;  \bar{\etav} \leq 0,\; \lambdav \geq 0 \}
\end{array}
\end{eqnarray*eqnarray*}
The objective function $G$ is submodular in $(\bar{\etav},\bold k,\lambdav, \bar{\xiv})$ and the constraint set $B_1$  is a sublattice of $(\bar{\etav},\lambdav)$.  Since minimizing submodular function over a sublattice preserves submodularity, $f$ is submodular in $(\bold{k}, -\xiv)$.  \qed
}

We first characterize the structural properties of production decisions and then proceed to fully characterize the optimal solution.  Let us substitute $\bold y = \bold z+ \bold x$, denoting sales, see Figure \ref{figure:frameGeneral}.

Now $f(s,\bold{k},\xiv)$ in{ the production problem} can be
expressed as follows:
\begin{eqnarray}
    f(s,\bold {k},\xiv) &=& \max_{(\bold{y,x}) \in A_1(\bold{k},\xiv)} (p-c)y^d + s(p-c)y^o + (s-1)c(x^o -x^d) %\nonumber \\
    -t(x^o + x^d) \label{eqn:fSales}\\
    A_1(\bold{k},\xiv) &=& \{(\bold{y,x}):\; \xiv \geq \bold y \geq \bold x \geq 0, \; \; y^d \leq k^d+x^d -x^o, y^o \leq k^o+x^o-x^d\} \label{eqn:A1}
\end{eqnarray}

%The objective function in the above formulation has intuitive interpretation.  The first two terms, $(p-c)y^d$ and $s(p-c)y^o$, are profits as if all production were made locally without any transshipment.  The third term $(s-1)c(x^o -x^d)$ is the relative production cost difference between two facilities.  The last term is the transshipment cost $t(x^d+x^o)$.
The first constraint in the constraint set $A_1$ states that sales must be greater than transshipped quantity but less than demand.  The last two constraints state that the sales are bounded by the sum of a facility's local capacity and net transshipment from the other facility.

Since local capacity plus net transshipment define available quantity to the local market, we immediately have:


{\lemma For a given feasible production decision $\bold x$, the optimal sales are $y^{d*} =(k^d + x^d-x^o)\wedge \xi^d$ and $y^{o*} = (k^o+x^o-x^d)\wedge \xi^o$.
\label{lemma:baseSales}}

For any feasible $\bold y^*$, we need $\bold x \leq \bold y^*$.  Thus, a transshipment problem can be expressed in terms of $\bold x$:
\begin{eqnarray}
    f(s,\bold {k},\xiv) &=& \max_{\bold{x} \in A_2(\bold{k},\xiv)} (s-1)c(x^o -x^d)-t(x^o + x^d)+ (p-c)((k^d + x^d-x^o)\wedge \xi^d) \nonumber\\
    && + s(p-c)((k^o+x^o-x^d)\wedge \xi^o)  \label{eqn:transship}\\
     A_2(\bold{k},\xiv) &=& \{\bold{x}:\;  \xiv \geq \bold x \geq 0, \;\; x^o  \leq k^d,\; x^d \leq k^o\} \label{eqn:A2}
\end{eqnarray}
The above problem can be interpreted as an assignment of capacity to locations.  Through the assignment, the new available quantities ideally reach the targets, {\it i.e.}, the demand in both markets.

To further characterize the optimal solution of the transshipment decisions, let $\bar{\bold k} = \bold k - \xiv$ and $\bar{K} = \bar{k}^d+\bar{k}^o$.  A positive (or negative) $\bar{\bold k}$ represents the capacity-overage (or capacity-shortage) for a given demand.  $\bar{K}$ is the total capacity net total demand.  Let $\bar{p} = p-c$, the profit margin of sales.  We also let $\bar{t}^d = (1-s)c-t$ and $\bar{t}^o = (s-1)c-t$.  $\bar{t}^d$ represents the relative cost difference of meeting domestic demand using overseas production, and $\bar{t}^o$ {\it vice versa.}


{\theorem Optimal transshipment decisions are $x^{d*} = (k^o\wedge \xi^d \wedge \bar{x}^d)^+$ and $x^{o*} = (k^d\wedge \xi^o \wedge \bar{x}^o)^+$, where
\begin{eqnarray}
\bar{x}^d= \left\{
  \begin{array} {ll}
  -\infty  \quad & \mbox {if $\bar{t}^d \leq -\bar{p}$}\\
  -\bar{k}^d  \quad & \mbox{ if $\bar{K} \geq 0$ and $0 \geq \bar{t}^d
  \geq -\bar{p}$, or $\bar{K} \leq 0$ and $ s\bar{p} \geq
  \bar{t}^d \geq (s-1) \bar{p}$} \\
\bar{k}^o  \quad  & \mbox{ if $\bar{K} \geq 0$ and $s\bar{p} \geq
\bar{t}^d \geq 0$, or $\bar{K} \leq 0$ and $  (s-1) \bar{p}\geq
  \bar{t}^d \geq -\bar{p}$}\\
  \infty \quad &\mbox {if $\bar{t}^d \geq s\bar{p}$}
  \end{array} \right.
\end{eqnarray}
\begin{eqnarray} \bar{x}^o= \left\{
  \begin{array} {ll}
  -\infty \quad & \mbox {if $\bar{t}^o \leq -s\bar{p}$}\\
  -\bar{k}^o \quad & \mbox{ if $\bar{K} \geq 0$ and $0 \geq
  \bar{t}^o
  \geq -s\bar{p}$, or $\bar{K} \leq 0$ and $ \bar{p} \geq
  \bar{t}^o \geq (1-s) \bar{p}$} \\
\bar{k}^d  \quad  & \mbox{ if $\bar{K} \geq 0$ and $\bar{p} \geq
\bar{t}^o \geq 0$, or $\bar{K} \leq 0$ and $  (1-s) \bar{p}\geq
  \bar{t}^o \geq -s\bar{p}$}\\
  \infty \quad &\mbox {if $\bar{t}^o \geq \bar{p}$.}
  \end{array} \right.
\end{eqnarray}
\label{pro:bSol}}

\proof See Appendix. \qed

The interpretation of Theorem \ref{pro:bSol} is quite intuitive.  The optimal transshipment quantity $x^d$ depends mostly on the relative cost differences of cross sales $ \bar{t}^d$.  The higher the cost differences, the higher the target transshipment level $\bar{x}^d$.  Since the marginal benefit changes only at $-\bar{k}^d$ and $\bar{k}^o$, {\it i.e.}, the domestic capacity shortage and the overseas capacity overage, the target transshipment level is one of these values.

Note that many papers (including Chowdhry and Howe \cite{Chowdhry1999}) assume complete capacity pooling, where all demands are to be satisfied as long as the total capacity is higher than or equal to the total demand.  Certainly this is not optimal if the exchange rate changes dramatically.  To see this, let us consider the following example.

\medskip

%%%% **** CANDIDATE FOR REMOVING ******

\noindent {\bf Example} {\em Let $k^d = 10,\; \xi^d = 2, \; k^o = 3,\; \xi^o = 5 $ and $s = 0.4, \; p = 10,\; c = 5$.  In this case, it is optimal to use all overseas capacity and no domestic capacity at all.  Two units of overseas capacity are used for domestic demand and one unit for overseas demand.  Adding any domestic production has negative marginal benefit because there are only two uses of domestic production: (1) meeting domestic demand and pushing overseas product to overseas demand, where the marginal benefit is $p-c + ps - p = -1$; and (2) meeting overseas demand; then the marginal benefit is also -1 because $ ps - c = -1$.}




\subsection{Properties of Financial Hedging}





\subsubsection{Bond Only}


We use two intuitive economic terms.  Let the effective risk tolerance $R_i =\sum_{j=0}^i (\rho_j(1+r_f)^{j-i})$ and the time weight present value equivalent $w^\prime_i = \sum_{j=0}^i(\rho_j \ln((1+r_f)^j \rho_j)/w_j)/(1+r_f)^{i-j})$.

The effective risk tolerance $R_n$ is the sum of discount risk tolerance in all periods from $n$ to $0$. It reflects the decision maker's ability to use bond investment spreading risk across time, he is effectively less risk averse to an earlier resolution of the cash flow uncertainty than a later one. This effective risk tolerance captures the inter-temporal risk preference that a decision maker prefers an earlier resolution of the uncertainties of the same cash flow.

The time weight present value equivalent $w^\prime_i$ is the present value of cash equivalent values of the time weights in all periods $j \in \{0, \cdots, i\}$.
The discount factor from period $j$ to $i$ is $1/(1+r_f)^{i-j}$, which is applied to the period $j$'s cash equivalent value of $w_j$. This value is $\rho_j\ln ((1+r_f)^j\rho_j)/w_j$ because $w_j = (1+r_f)^j \rho_j \exp(-(\rho_j\ln ((1+r_f)^j\rho_j/w_j))/\rho_j)$.


Using this definition, we first obtain the analytical solution of PCEV for a given single uncertain operational cash flow.

\begin{lemma} \label{lem:pcev-Xn-gen}
For a given initial wealth $\beta_n$ and a given uncertain operational cash flow with $f_n$ in period $n$ and with $0$ in all other periods, the optimal bond investment satisfies
\begin{eqnarray}\label{eqn:beta-Xn}
\frac{\beta_i + w_i^\prime}{R_i} = \frac{f_n+\beta_n + w_n^\prime}{R_n}, \; \forall 0\leq i< n;
\end{eqnarray}
the maximum utility is
\begin{eqnarray} \label{eqn:Un-Xn}
\mathcal{U}_n(f_n, 0, \cdots, 0|\beta_{n}) = \E[-R_n (1+r_f)^n \exp (-  \frac{f_n + \beta_n +w_n^\prime}{R_n})], \; \forall n \geq 0;
\end{eqnarray}
and the present certainty equivalent value is independent of $\beta_n$,
\begin{eqnarray}
\V_n(f_n,0,\cdots,0)=-R_n \ln\big{(} \E[\exp(-\frac{f_n}{R_n})]\big{)}, \; \forall n \geq 0. \label{eqn:pcev-Xn-gen}
\end{eqnarray}
\end{lemma}

\OUT{

\proof We prove (\ref{eqn:beta-Xn}) and (\ref{eqn:Un-Xn}) first, then (\ref{eqn:pcev-Xn-gen}) will follow. By (\ref{eqn:bondInv-rho}), the left side of (\ref{eqn:Un-Xn}) is
\begin{eqnarray*eqnarray*}
\max_{\beta_i, \forall 0 \leq i < n}\{\E[ -w_n \exp( - \frac{f_n + \beta_n - \beta_{n-1}/(1+r_f)}{\rho_n}) - \sum_{i=0}^{n-1}w_i \exp (-\frac{\beta_{i-1} -\beta_{i-1}/(1+r_f)}{\rho_i}) ]\}.
\end{eqnarray*eqnarray*}

To this maximization problem, we apply variable substitutions as follow:
\begin{eqnarray*eqnarray*}
\beta_i^{\prime} = \left\{ \begin{array}{ll}
f_n + \beta_n + w_n^\prime, & \mbox{if}\; i = n; \\
\beta_i + w_i^\prime, & \mbox{if} \;  0\leq i \leq n-1.
\end{array}
\right.
\end{eqnarray*eqnarray*}
This variable substitution transforms the maximization problem to be
\begin{eqnarray*}
\max_{\beta_i^{\prime}, \forall 0 \leq i < n}\{\E[ - \sum_{i=0}^{n}(1+r_f)^i\rho_i \exp (-\frac{\beta^\prime_{i} -\beta^\prime_{i-1}/(1+r_f)}{\rho_i}) ]\}.
\end{eqnarray*}
The first order necessary conditions are $[
\beta^\prime_1 - \beta^\prime_0/(1+r_f)]/\rho_1 = \beta^\prime_0/\rho_0$ for $i=0$, and for $0 < i \leq n-1$, $[\beta^\prime_{i+1} -\beta^\prime_{i}/(1+r_f)]/\rho_{i+1}  = [\beta^\prime_{i} -\beta^\prime_{i-1}/(1+r_f)]/\rho_i$. These equations imply that the optimal decisions satisfy $\beta^\prime_i/R_i = \beta_{i+1}^\prime/R_{i+1}$ for all $ 0\leq i\leq n-1$. As a result, the risk adjusted consumption cash in all periods are equalized at
\[ \frac{\beta_n^\prime-\beta_{n-1}^\prime/(1+r_f)}{\rho_n} = \frac{\beta_n^\prime-\beta_{n}^\prime R_{n-1}/[R_n(1+r_f)]}{\rho_n}  = \frac{\beta_n^\prime}{R_n},\]
 where the last equation follows from $R_{n} = R_{n-1}/(1+r_f) + \rho_n$. As a result, the maximum utility is $\E[-(1+r_f)^nR_n \exp (- \frac{\beta^\prime_n}{R_n} )]$. We obtain (\ref{eqn:beta-Xn}) and (\ref{eqn:Un-Xn}) by substituting back the original decision variables. Finally, the PCEV's independence on $\beta_n$ and  (\ref{eqn:pcev-Xn-gen}) follow from (\ref{eqn:PCEV}).

}

\proof See Appendix. \qed


The optimal investment in (\ref{eqn:beta-Xn}) implies that the decision maker spreads cash consumption, adjusted by the time weight present value equivalent $w_i^\prime$, across time in proportion to the effective risk tolerance.  As a result, the decision maker {\em equalizes} every period's discounted utility and her total expected utility then becomes (\ref{eqn:Un-Xn}). This maximum utility is the sum of utilities discounted by risk tolerance adjusted discount rate because $R_n(1+r_f)^n = \sum_{j=0}^n \rho_j(1+r)^{-j}$. The formula (\ref{eqn:pcev-Xn-gen}) shows that the PCEV depends on the effective risk tolerance, but not on the initial wealth, because intuitively adding a certain cash, e.g., initial wealth and the time weight present value, does change the PCEV of an uncertain operational cash flow.

We next extend these intuitions in the PCEV evaluation of a single period cash flow to any cash flows so that we have a clear understanding of the optimal bond investment strategy and general PCEV evaluation.

{\lemma \label{lem:pcev-bond} For a general cash flow with operational cash flow $f_i$ in each periods for all $0\leq i \leq n$ and initial wealth $\beta_n$, then for all $n \geq 1$, the present certainty equivalent value is independent of $\beta_n$,
\begin{eqnarray}\label{eqn:pcev-gen}
\V_n(f_n, \cdots, f_0) = \V_n(f_n+\frac{\V_{n-1} (f_{n-1}, \cdots, f_0)}{1+r_f}, 0, \cdots, 0);
\end{eqnarray}
the optimal bond investment satisfies
\begin{eqnarray}\label{eqn:beta-gen}
\frac{\beta_{n-1} + w_{n-1}^\prime+\V_{n-1}(f_{n-1},\cdots,f_0)}{R_{n-1}} = \frac{f_n+\beta_n+ w_n^\prime -(\beta_{n-1} + w_{n-1}^\prime)/(1+r_f)}{\rho_n};
\end{eqnarray}
and the maximum utility is
\begin{eqnarray} \label{eqn:Un-gen}
\mathcal{U}_n(f_n,  \cdots, f_0|\beta_{n}) = \E[-R_n (1+r_f)^n \exp (-  (f_n + \beta_n + \frac{\V_{n-1}(f_{n-1},\cdots, f_0)}{1+r_f}+ w_n^\prime)/R_n)].
\end{eqnarray}
 }

These results are extension of the results of Lemma~\ref{lem:pcev-Xn-gen}. Similarly, the optimal investment (\ref{eqn:beta-gen}) means that the risk tolerance and time weight adjusted consumption cash are spread equally across periods. As a result, the maximum utility (\ref{eqn:Un-gen}) is the sum of utilities discounted by risk tolerance adjusted discount rate.
The PCEV (\ref{eqn:pcev-gen}) implies that the PCEV can be computed by adding discounted future PCEV to the current uncertain cash flow, and the discount rate depends on the risk free interest rate only. Using (\ref{eqn:pcev-gen}) and (\ref{eqn:pcev-Xn-gen}), we can evaluate the PCEV of any given uncertain operational cash flow when bond investment is allowed to improve the consumption utility.





\subsubsection{Financial Instruments}

If an operational cash flow has cash in period $n$ only and 0 in all other periods, the derivative investment reduces to bond investment. Consequently, Lemma~\ref{lem:pcev-Xn-gen} remains true under financial hedging.

We now develop a recursive equation for evaluating PCEV of any general cash flows with financial hedging.

{\lemma \label{lem:pcev-FH}
For a general cash flow with operational cash flow $f_i$ in each periods for all $0\leq i \leq n$ and initial wealth $\beta_n(s_n)$, then for all $n \geq 1$, the present certainty equivalent value is independent of $\beta_n(s_n)$,
\begin{eqnarray}\label{eqn:pcev-FH}
\V_n(f_n, \cdots, f_0) = \V_n(f_n+\frac{\E[\V_{n-1} (f_{n-1}, \cdots, f_0)|S_n=s_n]}{1+r_f}, 0, \cdots, 0);
\end{eqnarray}
the optimal derivative investment satisfies
\begin{eqnarray}\label{eqn:beta-FH}
\frac{\beta_{n-1}(s_{n-1}) + \V_{n-1}(f_{n-1},\cdots,f_0|s_{n-1}) + w_{n-1}^\prime}{R_{n-1}} = \frac{f_n+\beta_n(s_n)-\frac{\E[\beta_{n-1}(S_{n-1})|S_n=s_n]}{1+r_f}+ w_n^\prime}{\rho_n};
\end{eqnarray}
and the maximum utility is
\begin{eqnarray} \label{eqn:Un-FH}
\mathcal{U}_n(f_n,  \cdots, f_0|\beta_{n}(s_n)) = \E[-R_n (1+r_f)^n \exp (-  \frac{f_n + \beta_n + \frac{\E[\V_{n-1}(f_{n-1},\cdots, f_0)|S_n=s_n]}{1+r_f}+ w_n^\prime}{R_n})].
\end{eqnarray}
 }

The results under derivative investment are different from the ones under bond investment, which is due to the optimal derivation investment (\ref{eqn:beta-FH}).  This optimal derivative implies not only that the adjusted consumption cash are equalized across periods, which the optimal bond implies too, but also that the adjusted consumption are equalized across realization of exchange rate in any given periods. This latter equalization further improves the decision maker's present certainty equivalent value by adding, to the current uncertain cash flow, the expected future PCEV discounted at risk free rate, as shown by (\ref{eqn:pcev-FH}). Using (\ref{eqn:pcev-Xn-gen}) and (\ref{eqn:pcev-FH}), we can evaluate PCEV of any operational cash flow when financial hedging is used to improve consumption utility.

{\theorem \label{cor:fwd}
If the exchange rate follows a birth-death process (i.e., goes either one state up or one state down), then the optimal financial hedging is to use forward contract and risk-free bond.}









\subsection{Properties of Operational Hedging}



Having characterized the properties and optimal solutions of the production and transshipment decisions, we now proceed to study the properties of the optimal capacity decision that maximizes MRN's PCEV. While the maximum utility is joint concave in domestic and overseas capacity allocation, it is not obvious whether the PCEV is also concave.  The following lemma is helpful to derive the PCEV concavity property.
\begin{lemma} \label{lem:ln-convex}
If $f_i(x)$ is concave and $w_i\geq 0$ for all $i$, then $\ln(\sum_{i}w_i\exp[-f_i(x)])$ is convex in $x$. 
\end{lemma}

\proof Note that for any function $h(x)$, $\ln [h(x)]$ is convex in $x$ if and only if $h''(x)h(x) - [h'(x)]^2 \geq 0$. Let $h(x) = \sum_{i}w_i\exp[-f_i(x)]$, we can verify that $h''(x)h - [h'(x)]^2
= - \sum_{i} w_i^2\exp[-2f^i(x)]f_i^{''}(x) + \sum_{ij, i\neq j}w_i w_j \exp [-(f_i(x)+ f_j(x))][( f^{'}_i(x) - f^{'}_j(x))^2 - f_i^{''}(x) - f_j^{''}(x)]$, which is nonnegative because of $f_i$'s concavity. \qed

\begin{theorem} \label{the:baseCap}
The objective function in (\ref{eqn:capacityAlloc}) is concave in $k^d$.
\end{theorem}

\proof Since $f_i$ is increasing in $\bold{k}$, at optimal $k^d+k^o = K$. Since $f_i$ is joint concave in $\bold{k}$, $f_i$ is concave in $k^d$. We now show by induction that $\V_i(\cdot)$ is concave in $k^d$. First by (\ref{eqn:pcev-Xn-gen}), $\V_0(f_0)= -R_0\ln \big{(}\E [\exp(-f_0/R_0)]\big{)}$, which is concave in $k^d$ by Lemma \ref{lem:ln-convex}.

Suppose $\V_{n-1}(\cdot)$ is concave in $k^d$, then $f_n + \V_{n-1}(\cdot)/(1+r_f)$ is concave in $k^d$. Thus by (\ref{eqn:pcev-Xn-gen}), Lemma \ref{lem:ln-convex} implies that $\V_n(\cdot)$ in (\ref{eqn:pcev-gen}) is concave in $k^d$. Since expectation does not change concavity, $\V_n(\cdot)$ in (\ref{eqn:pcev-FH}) is also concave in $k^d$. \qed


%It suffices to show that this maximization problem is equivalent to a problem of maximizing a concave function of $\bold{k}$ in that they have the same optimal $\bold{k}$. Since the utility function is increasing in consumption cash, (\ref{eqn:bondInv-rho}) and (\ref{eqn:Un-Xn}) implies that $\mathcal {U}_n(\V_n(f_n, \cdots, f_0), 0, \cdot, 0)$ is increasing in $\V_n(f_n, \cdots, f_0)$. As a result, maximizing $\V_n(f_n,\cdots, f_0)$ is equivalent to maximizing $\mathcal {U}_n(\V_n(f_n, \cdots, f_0), 0, \cdot, 0)$, and because of (\ref{eqn:PCEV}) it is equivalent to maximizing $\mathcal {U}_n(f_n, \cdots, f_0)$.
%
%$\mathcal {U}_n(f_n, \cdots, f_0)$ is indeed jointly concave in $\bold{k}$. In its expression (\ref{eqn:bondInv-rho}), $f_i$ is concave in $\bold{k}$ by (\ref{lemma:conProfit}) and $-w_i \exp(-(f_i+\beta_{i}-\beta_{i-1}/(1+r-f))/\rho_i) $ is thus concave in $\bold{k}$ because an increasing concave function of a concave function is also concave. Linear combination and expectation of concave functions are concave. Finally maximization of a concave function is concave. The same argument applies to the exchange rate derivative investment.
%
%



So far we have completely characterized structural properties and optimal solutions of the Basic Model.  We showed that the problem behaves in an intuitive way with concave and submodular production and transshipment decisions.  The optimal financial hedging contracts (\ref{eqn:beta-FH}) which we identified, spread cash consumption across time and across different realization of exchange rate. The results also set the stage for a very efficient identification of optimal capacity allocation decisions.

In this section, we assumed that the price is fixed.  In many situations, price is a decision variable used to influence demand.  In the following section we study how allowing to change the price, in addition to production and transshipment, influences the optimal policy.
%{\pro If $\frac{\partial f(x_i,\bold {k, \xi})}{\partial
%k^o}-\frac{f(x_i,\bold {k, \xi})}{\partial k^d} \geq 0$, then
%$W_n$ is submodular in $\bold k$. \label{prop:tech1}}





\section{Price Model \label{sect:priceModel}}

Price is a powerful instrument because it allows a firm to increase profit by decreasing effective demand, and thus avoid unsatisfied demand when capacity is insufficient and, {\it vice versa}, increase demand when capacity is underutilized.  We model the demand as a linear function of price $\bar{\xi}^i = \xi_i - bp_i$.  The capacity decision and financial hedging decision are the same as those in the Basic Model (Section \ref{sect:basicModel}).  The difference is in the third decision.  Instead of production and transshipment decisions, we make joint price, production, and transshipment decisions.  These decisions are made after demand is revealed.  Thus, as before, the total quantity of production and transshipment does not exceed demand at a given price.
\OUT{
The problem can be formulated as follows:
\begin{eqnarray}
f(s,\bold{k,\xi}) &=& \max_{(\bold p, \bold y, \bold x) \in
B(\bold{k},\xiv)} J(\bold p, \bold y,\bold x) \label{eqn:fPrice}\\
J(\bold {p,y,x}) &=& (p^d-c)y^d + s(p^o-c)y^o + (s-1)c(x^o -x^d)\nonumber\\&& -t(x^o + x^d) \\
B(\bold{k},\xiv)& = &\{(\bold {p,y,x}): \; \bold p \geq 0,\;\bold
y \geq 0, \; \bold x \geq 0,\;  \bold y\leq \xiv-b\bold p, \nonumber \\
&& y_i \leq k_i+x_i-x_{j\neq i} \; \forall i \in \{o,d\} \forall j
\in \{o,d\}\}
\end{eqnarray}



The price and production decision problem cannot be solved independently, as choice of price determines the feasible region for production and transshipment decisions.  Taking advantage of the fact that it is always possible to avoid over-production, we further simplify the formulation by noting that

}
Since for any case when shortage takes place, it is beneficial to increase price, {\em optimal} decisions are always market-clearing:


{\lemma Considering only combinations of price and production/transshipment decisions satisfying $\xiv-b\bold p\geq \bold y \geq 0$, the optimal decisions satisfy $\bold p = (\xiv-\bold y)/b$. \label{lemma:priceSol}}


%\proof Clearly $J(\bold {p,y,x})$ increases in price and, thus, optimal $p$ is at the upper bound. \qed

Lemma \ref{lemma:priceSol} states that the optimal price is set to sell all products.  It transforms a {price, production, and transshipment decision} problem into a production/transshipment-decision-only problem:
\begin{eqnarray}
f(s,\bold {k},\xiv) &=& \max_{(\bold{y,x}) \in A_1(\bold{k},\xiv)}
G(\bold{x,y},\xiv) \label{eqn:fPSales}\\
G(\bold{x,y},\xiv) &=& (\xi^d-bc-y^d)y^d/b + s(\xi^o-bc-y^o)y^o/b
\nonumber \\&& + (s-1)c(x^o -x^d) -t(x^o + x^d)
\label{eqn:GPrice},
\end{eqnarray}
where $A_1(\bold{k},\xiv)$ is defined in (\ref{eqn:A1}).  This simplified problem has structure similar to (\ref{eqn:fSales}).  The first two terms of its objective function $G$ represent the profit if no transshipment were allowed, and the last two represent the benefit of transshipment.  This formulation leads to the following property.


{\lemma $f$ is non-decreasing and concave in $(\bold k, \xiv)$. \label{lemma:priceFcon}}


\proof The proof follows the same logic as that of Lemma \ref{lemma:conProfit}, where the constraint set $A$ is replaced by $A_1$ and objective function $J$ is replaced by $G$.  The proof hinges on concavity of $G$ in $(\bold x, \bold y, \xiv)$, which is true because $G$ is a negative quadratic function. \qed \OUT{}

Lemma \ref{lemma:priceFcon} allows us, later, to claim concavity of both { financial hedging decision} and { capacity decision} problems.  Building on the latest reformulation, sales can be explicitly expressed.  Let us define first $\bar{\bold y} \equiv (\xiv-bc)/2$:


{\lemma Let $\bold x \leq \xiv$.  Considering $\bold x \leq \bold y \leq \xiv, \; y^d \leq k^d+x^d - x^o, \; y^o \leq k^o -(x^d - x^o)$, the optimal sales are $y^{d*} = (x^d\vee \bar{y}^d)\wedge (k^d +x^d-x^o)$ and $y^{o*} = (x^o\vee \bar{y}^o)\wedge( k^o+x^o-x^d)$. \label{lemma:solPSales}}

\proof The objective function $G$ is concave in $\bold y$.  Thus, the unconstrained optimal $\bold y = \bar{\bold y}$.  Therefore, the optimal constrained solution can be expressed as $y^{d*} = (x^d\vee \bar{y}^d)\wedge \xi^d \wedge (k^d+x^d-x^o)$ and $y^{o*} = (x^o\vee \bar{y}^o)\wedge \xi^o \wedge (k^o+x^o-x^d)$. \qed

Lemma \ref{lemma:solPSales} states that $\bar{\bold y}$ serve as optimal sales target levels, which are independent of the transshipment decisions $\bold x$, and that actual sales will be limited by the constraints in the capacity and transshipment.  To further explore structural properties, optimal sales are substituted into the objective function of the problem.  With definition $\bar{t}^d \equiv (c-cs-t)b/2$ and $\bar{t}^o \equiv (cs-c-t)b/(2s)$, the cost difference of cross sales due to transshipment (these definitions differ slightly from those in the previous section), the price and production/transshipment decision problem is simplified to a transshipment-decision-only problem:
\begin{eqnarray}
f(s,\bold{k},\xiv) &=& [(\bar{y}^d)^2 + s(\bar{y}^o)^2 +
\max_{\bold {x} \in
A_2(\bold k, \xiv)} G(\bold x)]/b \label{eqn:fPNewTran}\\
G(\bold x) &=& 2\bar{t}^d x^d - ((x^d - \bar{y}^d)^+)^2 - ((k^d
-\xi^d +x^d-x^o)^-)^2 \label{eqn:GNew} \\ &&+ s(2\bar{t}^ox^o-
((x^o - \bar{y}^o)^+)^2-((k^o-\xi^o -x^d+x^o)^-)^2), \nonumber
\end{eqnarray}
where $A_2$ is defined in (\ref{eqn:A2}).  Since in the optimal transshipment decisions $x^d$ and $x^o$ are not positive at the same time, we have:
\begin{eqnarray}
f &=& [(\bar{y}^d)^2 + s(\bar{y}^o)^2 + \max(f^d(s,\bold k, \xiv),
sf^o(s,\bold k, \xiv))]/b\\
f^d(s,\bold k,\xiv)& =& -s((-\bar{y}^o)^+)^2) +\max_{(\xi^d\wedge
k^o) \geq x^d \geq 0} 2G^d( x^d ) \label{eqn:fd} \\G^d(x^d) &=&
\bar{t}^d x^d - ((x^d+k^d -\bar{y}^d )^-)^2/2 - ((x^d -
\bar{y}^d)^+)^2/2 \nonumber \\&&- s((x^d-k^o+\bar{y}^o)^+)^2/2.
\label{eqn:Gd}
\\f^o(s,\bold k,\xiv)& =& -((-\bar{y}^d)^+)^2) +\max_{(\xi^o\wedge
k^d) \geq x^o \geq 0} 2G^o( x^o ) \label{eqn:fo} \\G^o(x^o) &=&
\bar{t}^o x^o - ((x^o+k^o-\bar{y}^o )^-)^2/2 - ((x^o -
\bar{y}^o)^+)^2/2 \nonumber \\&&- ((x^o-k^d
+\bar{y}^d)^+)^2/2s.\label{eqn:Go}
\end{eqnarray}
This leads to the following properties.


{\lemma $f$ is submodular in $(\bold k, -\xiv)$.
\label{lemma:priceFsub}}

\proof See Appendix. \qed

Lemma \ref{lemma:priceFsub} formally states that the capacities and demands are complements and they are substitutes within themselves, even when price is a decision variable, mirroring Lemma \ref{lemma:subProfit} in the Basic Model.  The submodularity leads to similar properties in the financial hedging decisions.

To characterize the optimal decision, let us define $\bar{\bold k} \equiv \bold k-\bar{\bold y}$, the capacity-overage (if positive) or capacity-shortage (if negative) at two locations, and $\bar{K} \equiv \bar{k}^d + \bar{k}^o$, the total-capacity-overage (if positive) or the total-capacity-shortage (if negative).  With the redefinitions above, the optimal transshipment decisions are as follows:


{\theorem The optimal solutions are $x^{d*} = (k^o\wedge \xi^d \wedge \bar{x}^d)^+$ and $x^{o*} = (k^d\wedge \xi^o \wedge \bar{x}^o)^+$, where
\begin{equation}
\bar{x}^d = \left\{
  \begin{array} {ll}
  \bar{t}^d -\bar{k}^d \quad & \mbox{ if $\bar{t}^d \leq \bar{K}\leq 0$, or $\bar{K}\geq 0$ and $\bar{t}^d \leq 0$  } \\
(\bar{t}^d  - \bar{k}^d +s\bar{k}^o)/(1+s)  \quad  & \mbox{if $
\bar{K} \leq  \bar{t}^d \leq -s\bar{K}$ and $\bar{K}\leq 0$ }\\
\frac{\bar{t}^d  + \bar{y}^d1_{\bar{k}^o \geq
\bar{y}^d}+s\bar{k}^o1_{\bar{k}^o \leq \bar{y}^d}}{1_{\bar{k}^o
\geq \bar{y}^d}+s1_{\bar{k}^o \leq \bar{y}^d}} \quad  & \mbox{if
$0 \leq \bar{t}^d \leq (s+1)(\bar{y}^d\vee \bar{k}^o)-\bar{y}^d -
s\bar{k}^o$ and $\bar{K}\geq 0$}\\
(\bar{t}^d  +s\bar{k}^o)/s \quad  & \mbox{if $ -s\bar{K} \leq
\bar{t}^d \leq  s(\bar{y}^d - \bar{k}^o)$ and $\bar{K}\leq 0$} \\
(\bar{t}^d  + \bar{y}^d+s\bar{k}^o)/(1+s) \quad  & \mbox{if $
s(\bar{y}^d - \bar{k}^o) \leq \bar{t}^d $ and $\bar{K}\leq 0$}, \\
\quad  & \mbox{or $ (s+1)(\bar{y}^d\vee \bar{k}^o)-\bar{y}^d -
s\bar{k}^o \leq \bar{t}^d $ and $\bar{K}\geq 0$}, \end{array}
\right. \label{eqn:barXd}
\end{equation}
\begin{equation}
\bar{x}^o = \left\{
  \begin{array} {ll}
  \bar{t}^o -\bar{k}^o \quad & \mbox{ if $\bar{t}^o \leq \bar{K}\leq 0$, or $\bar{K}\geq 0$ and $\bar{t}^o \leq 0$  } \\
(\bar{t}^o  - \bar{k}^o +s'\bar{k}^d)/(1+s')  \quad  & \mbox{if $
\bar{K} \leq  \bar{t}^o \leq -s'\bar{K}$ and $\bar{K}\leq 0$ }\\
\frac{\bar{t}^o  + \bar{y}^o1_{\bar{k}^d \geq
\bar{y}^o}+s'\bar{k}^d1_{\bar{k}^d \leq \bar{y}^o}}{1_{\bar{k}^d
\geq \bar{y}^o}+s'1_{\bar{k}^d \leq \bar{y}^o}} \quad  & \mbox{if
$0 \leq
\bar{t}^o \leq (s'+1)(\bar{y}^o\vee \bar{k}^d)-\bar{y}^o - s'\bar{k}^d$ and $\bar{K}\geq 0$}\\
(\bar{t}^o  +s'\bar{k}^d)/s' \quad  & \mbox{if $ -s'\bar{K} \leq
\bar{t}^o \leq  s'(\bar{y}^o - \bar{k}^d)$ and $\bar{K}\leq 0$} \\
(\bar{t}^o  + \bar{y}^o+s'\bar{k}^d)/(1+s') \quad  & \mbox{if $
s'(\bar{y}^o - \bar{k}^d) \leq \bar{t}^o $ and $\bar{K}\leq 0$}, \\
\quad  & \mbox{or $ (s'+1)(\bar{y}^o\vee \bar{k}^d)-\bar{y}^o -
s'\bar{k}^d \leq \bar{t}^o $ and $\bar{K}\geq 0$}, \end{array}
\right. \label{eqn:barXo}
\end{equation}
and $s'=1/s$.
\label{theo:priceoptimal}
}

\proof See On-line Appendix. \qed \OUT{ the transshipment decision problem is represented by Eqs. (\ref{eqn:fo}) and (\ref{eqn:Go}).  The optimal $x^o$ satisfies the first-order-condition of $Go$:
\begin{equation}
G'^o(x^o) = \bar{t}^o- (x^o+\bar{k}^o )^- -(x^o - \bar{y}^o)^+
-s'(x^o-\bar{k}^d)^+=0, \label{eqn:XoFirst}
\end{equation}
which is illustrated in Figure \ref{fig:xoPrice}.
\begin{figure} [ht]
\begin{center}
\epsfig{file=xoDerivePriceModel.eps,width=4.5in,height=4.0in}
\end{center}
\caption{Illustration of $G'^o(x^o)$}\label{fig:xoPrice}
\end{figure}
$\bar{x}^o$ is obtained after $G'^o(x^o)$ is analyzed in the same
way as $G'^d(x^d)$. }

In summary, for Price Model, structural properties (concavity and submodularity), similar to those in Basic Model, are proved for all decisions.  In addition, the optimal price is always market clearing.  The optimal sales targets are independent of production and transshipment.  Sales targets and total profits decrease in price sensitivity.
Having characterized the properties of optimal price and production and transshipment decisions, concavity of Lemma \ref{lemma:priceFcon} implies that Theorem~\ref{the:baseCap} continues to hold, i.e., there is a unique optimal {capacity decision}.


As the structural properties and optimal solutions have been completely characterized for both the Basic Model and the Price Model, they allow us to efficiently perform numerical study. \OUT{





\section{Extension \label{sect:exension}}

We discuss now the following extensions:

(1) impact to capacity decision caused by optimal hedging vs linear hedging or no hedging.

(2) correlation among demands

(3) correlation between demand and exchange rate

(4) carry inventory across periods.
\\
$W_n$ is the total utility of the firm for $n$ periods to go, the functional form is below:
\begin{eqnarray}
W_n( x,s_n, \bold k) &=& \max_{\E [\rho(s_{n-1})|s_n]=0}
\E_{s_{n-1}|s_n} U_n (x, s_{n-1}, \rho (s_{n-1}),\bold k) \label{wfun}\\
U_n(x,s_{n-1},\rho(s_{n-1}), \bold k )&=& E_{\bold
\xi}\max_{(\bold a, \bold y)
\in A (x,\bold {k, \xi})} [J_n (\bold {a,y}, s_{n-1},\rho(s_{n-1}),\bold{ \xi,k})]  \\
J_n (\bold{a,y},s_{n-1},\rho(s_{n-1}), \bold {\xi,k}) &=&
g_n(\bold{ a, y }, s_{n-1},\rho(s_{n-1}),\bold \xi) \nonumber \\&&
+ W_n((x+a^d+a^o-\xi^d-\xi^o)^+, s_{n-1}, \bold k)\label{vfun}
\\ g_n(\bold{ a, y}, s_{n-1},\rho(s_{n-1}),\bold \xi) &=& u(ps_{n-1} y^o + py^d-ca^d - cs_{n-1}a^o \nonumber \\ &&- h (x + a^d + a^o + - \xi^d-\xi^o)^+ + \rho(s_{n-1}) )\\
A(x,\bold{k,\xi}) &=& \{\bold k \geq \bold a \geq 0, \bold \xi \geq \bold y \geq 0, x+a^d + a^o \geq y^d + y^o\}\\
s_{n-1} &=& s_n(1 + \epsilon) \nonumber \\ W_0(x, s_0) &=& 0
\nonumber
\end{eqnarray}
\OUT{ \left\{
  \begin{array} {l}
  -c\min(k^d, \xi^d+\xi^o) - s_{n-1}c\min (k^o, (\xi^d+\xi^o -
  k^d)^+) + p\xi^d + s_{n-1} p \xi^o \\
   \quad \mbox{ if $s_{n-1}\geq 1 $  and $\xi^d + \xi^o \leq k^d + k^o$} \\
  -s_{n-1} c \min (k^o, \xi^d+\xi^o) - c \min (k^d, (\xi^o+\xi^d -
  k^o)^+) + p\xi^d + s_{n-1}p\xi^o \\
   \quad \mbox{if $s_{n-1}\leq 1 $  and $\xi^d + \xi^o \leq k^d + k^o$} \\
  -ck^d - cs_{n-1}k^o + ps_{n-1} \min(\xi^o, k^d + k^o) +
   p \min (\xi^d, (k^d + k^o-\xi^o)^+) \\
    \quad \mbox{if $s_{n-1}\geq 1 $  and $\xi^d + \xi^o \geq k^d + k^o$} \\
  -ck^d - cs_{n-1}k^o + p \min(\xi^d, k^d + k^o) +
   ps_{n-1} \min (\xi^o, (k^d + k^o-\xi^d)^+) \\
    \quad \mbox{if $s_{n-1}\leq 1 $  and $\xi^d + \xi^o \geq k^d + k^o$} \\
  \end{array} \right.       \label{gfun}
\\ s_{n-1} &=& s_n(1 + \epsilon), \nonumber \\ W_0(\bold x, s_0) &=& 0. \nonumber
\end{eqnarray}
}
where $\epsilon$ is the random factor in exchange rate.  In above formulation, the objective in (\ref{wfun}) is to maximize utility of the remaining $n$ periods.  Production and meeting demand decision is made after realization of random demand and exchange rate.

}



\section{Numerical Study \label{sect:numer}}

The structural results (Sections \ref{sect:basicModel} and \ref{sect:priceModel}) enable us to perform an extensive numerical study and to answer the question: What are the favorable conditions for either financial or operational hedging (or both)?  In the case we describe in biggest detail, we assume that (a) we sell both in domestic and in overseas market.  We also evaluate the case (b) where we sell only overseas.  In (a) and (b) price is exogenous.  Finally, we look at case (c), where we sell both in domestic and overseas market and the price is endogenous.  To make a fair evaluation of financial, operational, and joint (financial plus operational) hedging, their relative efficiency is compared to the base case.  The base case (whether it is (a), (b), or (c)) always assumes that all capacity is located in the domestic facility and no financial hedging is used.  Operational hedging allows part of the capacity to be located overseas, while financial hedging keeps all capacity in domestic location, but allows unconstrained use of financial instruments.  The efficiency of other models is expressed as a percentage of increase over the PCEV of the base case.  

We use bionomial tree to model exchange rate evoluation.  It changes with rate {\it $(1+\epsilon)$} up and $1/{\it (1+\epsilon)}$ down in each period with mean equal to the exchange rate at the beginning of the period.  To make computation time manageable, maximum exchange rate does not exceed $(1+\epsilon)^3$ and minimum exchange rate does not exceed $(1+\epsilon)^{-3}$.  Demands in each market are modelled as Erlang distribution with mean of 5.  Demands and exchange rates are independent. 

Since parameters of the model may significantly change the relative strength and weakness of financial hedging and operational hedging, a wide range of parameters has been chosen for the numerical study. We vary five parameters, listed below, across four values each, resulting in $1,024$ $(4^5)$ instances.  The values of parameters are: total available capacity $K= \{6, 8, \bold{10}, 12\}$, the risk-tolerance factor $\rho = \{10, 12.5, \bold{15}, 17.5, 20\}$, the sales price $p = \{6, 8, \bold{10}, 12\}$, the exchange rate parameter $\epsilon=\{0.01, \bold{0.1},$ $0.3,$ $0.5 \}$, and the demand coefficient of variation ranges in $\{10\%, 20\%, \bold{33.3\%}, 50\%\}$.  The production cost $c=5$ is kept unchanged because relative effects are revealed through the change of the price and transportation cost. All experiments run for 20 periods.  Except when we explicitly state it, all experiments assume 0 transportation costs.  Transportation costs favor strongly operational hedging, as we show in one of later graphs.



%???
%For a portion of the study, where we examine the effects of the length of planning horizon, differences in starting exchange rate, and long-term effect of financial hedging contracts, we use {\em central} values which are emphasized in bold above.  For that portion of the study, we have (a) planning horizon ranges from 1 to 100; (b) the starting exchange rate ranges from $1.1^{-3}$ to $1.1^3$.  \OUT{ and (c) the maximum length of the financial hedging contract ranges from 1 to 20.  }





\subsection{Operational Hedging {\it versus} Financial Hedging}

Our numerical study suggests that both operational and financial hedging are very effective in case (b) when we sell only in overseas market, with operational hedging being somewhat better on average (61\%) than financial one (51\%) and with joint hedging providing 70\% benefits.  The effectiveness, while still significant, decreases when we sell both in domestic and overseas markets (case (a).  In this situation, the difference between operational and financial hedging is much more significant.  On average, the savings realized from operational hedging, amount to about 6.8\% {\it vs.} \hspace{-0.07in}1\% from financial hedging.  The incremental benefit of financial hedging is around 0.2\%.  In our Price Model (case (c)), all benefits are much smaller.  On average the savings from operational hedging amounted to 1.4\% {\it vs.} 0.22\% from financial hedging, the incremental benefit of financial hedging is about 0.2\% - almost the same as in our Basic Model.  The exchange rate coefficient of variation has to be very high (as much as 0.5) for financial hedging to dominate operational hedging.  Note that in case (b), all revenues are overseas while all costs are domestic.  Thus, financial hedging influences a very significant portion of profit.  Similarly operational hedging, can ``match'' most of the revenues by allocating production to overseas market.  In case (a), where we sell in both domestic and overseas market, in most situations, domestic production is used anyhow to satisfy domestic demand, and financial hedging influences only part of the revenues.  Operational hedging, however, has a similar increment of benefits as case (b): it can not only on average match costs with revenues, but also shift production back and forth to cheaper production site.  Clearly, in Price Model, using prices reduces the overall variance of cash flow, thus making hedging less needed and less effective.



The relative efficiencies of financial and operational hedging are affected by numerous other parameters.  These effects are summarized below.  Unless specified otherwise, the discussion applies to selling in both markets with exogenous price (case (b)) and endogenous price (case (c)).


{\noindent \em Total available capacity}.  The capacity effects are illustrated in Figure \ref{fig:capacity}.  Both financial hedging and operational hedging benefits increases when capacity increases.  Note that the higher capacity implies more flexibility for operational hedging to take advantage of it, while financial hedging is mostly not influenced by it.

\begin{figure}[ht]
\begin{center}
\begin{minipage}{6in}
    \begin{minipage}{3.1in}
        \epsfxsize=2.8in
   \hspace{-0.0in}     \epsfbox{capExoDmd.eps}
    \end{minipage}
    \begin{minipage}{2.8in}
        \epsfxsize=2.8in
    \hspace{-0.0in}    \epsfbox{capEndDmd.eps}
    \end{minipage}
\end{minipage}
\vspace{.05in} \caption{Relative savings as function of total capacity in the Basic Model (left) and in the Price Model (right).} \label{fig:capacity} \vspace{-.2in}
\end{center}
\end{figure}



{\noindent \em Risk-averse factor.} The risk-averse effects are illustrated in Figure \ref{fig:risk}.  The savings increase and then decrease.  When the risk-averse factor is small, the utility function is almost linear.  Thus, reducing variability of cash flow does not help and financial hedging has negligible benefits.  Note that operational hedging still benefits by using cheaper production.  Interestingly, similar logic applies when the exchange rate factor is very large, as the utility function becomes very flat (extra \$1,000 is not increasing utility significantly a high values).  For moderate risk-factors, clearly hedging can make a difference.  \begin{figure}[ht]
\begin{center}
\begin{minipage}{6in}
    \begin{minipage}{3.1in}
        \epsfxsize=2.8in
   \hspace{-0.0in}     \epsfbox{riskExoDmd.eps}
    \end{minipage}
    \begin{minipage}{2.8in}
        \epsfxsize=2.8in
   \hspace{-0.0in}     \epsfbox{riskEndDmd.eps}
    \end{minipage}
\end{minipage}
\vspace{.05in} \caption{Relative savings as function risk-averse factor in the Basic Model (left) and  in the Price Model (right).}
\label{fig:risk} \vspace{-.2in}
\end{center}
\end{figure}



{\noindent \em Sales price.} The price effects do not apply to case (c) and only illustrated for case (b), see the left of Figure \ref{fig:priceAndSens}.  Both operational and financial hedging relative benefits decrease with the sales price.  When the price increases, the absolute benefits increase, but at a slower rate than that of the base-case.  Significant factor is simply fast increase of the utility of the base case.  \OUT{Additionally, for financial hedging, this is due to the convexity of utility function's derivative.  Note that financial hedging's role is to bring utilities closer to each other under all exchange rate scenarios.  Let us consider two cash-flow scenarios.  Since cash flow is almost linear in price, the two cash flow can be expressed as $pC_1$ and $pC_2$ with probabilities $q_1$ and $q_2$, respectively.  Then, the financial hedging utility is $U(p(q_1C_1+q_2C_2))$ and the base case utility $q_1U(pC_1) + q_2U(pC_2)$.  When $U^\prime$ is convex (which is the case for exponential utility), the financial hedging utility increases slower than the base-case utility.}  The operational hedging savings increase slower because the savings apply only to a portion of the revenues.

\begin{figure}[ht]
\begin{center}
\begin{minipage}{6in}
    \begin{minipage}{3.1in}
        \epsfxsize=2.8in
    \hspace{-0.0in}    \epsfbox{priceExoDmd.eps}
    \end{minipage}
    \begin{minipage}{2.8in}
        \epsfxsize=2.8in
    \hspace{-0.0in}    \epsfbox{priceEndDmd.eps}
    \end{minipage}
\end{minipage}
\vspace{.05in} \caption{Relative savings as function of sales price in the Basic Model (left) and as function of price sensitivity in the Price Model (right).} \label{fig:priceAndSens}
\vspace{-.2in}
\end{center}
\end{figure}



{\noindent \em Sales price sensitivity.} Price sensitivity effects apply to the Price Model only (case (c)) and are illustrated in the right of Figure \ref{fig:priceAndSens}.  The savings of both financial and operational hedging increases in price sensitivity.  When price sensitivity increases, utilities of all models decrease.  The utilities of both operational and financial hedging decrease, however, slower than that of the base-case for the same reasons as above:  for financial hedging, this is caused by the convexity of exponential utility function's derivative, similar to the price effect, while for operational hedging, due to portion of profits not being influenced.



{\noindent \em Transportation cost.}  This is the only place where we allow consider positive transportation costs.  Transportation cost's effect
on operational hedging is opposite to that of  financial hedging,
as shown in Figure \ref{fig:trans}.  Operational hedging savings increase with transportation costs, but financial hedging savings slowly decrease.  Higher transportation costs result in less goods being transshipped and, thus, less foreign cash flow.  Therefore, financial hedging is less needed.
 Operational hedging optimally allocates a portion of total capacity overseas, which reduces transportation costs.  The higher the transportation cost, the greater the savings realized through operational hedging.  This effect is more pronounced in the Basic Model (case (b)) than in the Price Model (model (c)) since being able to change prices decreases volume of transshipment.

\begin{figure}[ht]
\begin{center}
\begin{minipage}{6in}
    \begin{minipage}{3.1in}
        \epsfxsize=2.8in
    \hspace{-0.0in}    \epsfbox{transExoDmd.eps}
    \end{minipage}
    \begin{minipage}{2.8in}
        \epsfxsize=2.8in
    \hspace{-0.0in}    \epsfbox{transEndDmd.eps}
    \end{minipage}
\end{minipage}
\vspace{.05in} \caption{Relative savings as function of
transportation cost in the Basic Model (left) and in the Price Model (right).} \label{fig:trans} \vspace{-.2in}
\end{center}
\end{figure}



{\noindent \em Exchange rate parameter.} The effect of exchange rate variance is illustrated in Figure \ref{fig:exch}.  Directly from our analytical results, financial hedging changes the linear average of the disutility to a geometric average.  Thus, the savings are higher for a larger variation of exchange rate.

The benefits due to operational hedging may either increase or decreases as a function of the exchange rate, depending on the transshipment cost.  The savings come from two sources: reduced exposures to exchange rate risks and reduced transportation costs.  While the savings from the reduced exposure always increase in the variance of exchange rate, the savings from the reduced transportation cost decrease due to capacity increasingly censoring the needed transshipment.  Since transportation always benefits operational hedging, we illustrate here only the situation when transportation costs are absent.  This dynamics is actually straightforward: higher variability of exchange rates allows financial hedging to capture a significant portion of that variability, while operational hedging results in production being shifted more often to cheaper location.\footnote{When exchange rate variance is small, operational hedging may eliminate a big portion of transshipment and save most of the transportation costs.  When exchange rate variance is high, operational hedging may not save the transshipment cost at all as it may be optimal to allocate all capacity to the domestic facility, as in the base case.}

%To see this, lets consider a simple example where $p=2, \; t =c=1, \; \xi^d = \xi^o = 5$ and $K= 10$. If exchange rate is 1 and constant, then operational hedging eliminate all transportation costs.  If exchange rate becomes 2 or 0 with half chance each, operational hedging allocates all capacity to domestic facility.  (For exchange rate of 2, domestic production is cheaper, and it is best to have all capacity domestic.  For exchange rate of 0, overseas production and prices are cheaper, but transportation cost rules out the advantage of overseas production.  Thus, all capacity is domestic.)

%The operational hedging savings decrease  in exchange rate variance at high transportation cost, increase in exchange rate
 variance at low transportation cost, both in a semi-concave way. The average operational savings exhibit semi-concave trend.

\OUT{ in the exchange rate variance, as shown in the left of Figure \ref{fig:exch}.  In the Price Model, however, the savings from high transportation costs are smaller, as shown in the right of Figure \ref{fig:trans}.  Thus, the average operational hedging savings increase in the variance of the exchange rate, as seen in the right of Figure \ref{fig:exch}. }
\begin{figure}[ht]
\begin{center}
\begin{minipage}{6in}
    \begin{minipage}{3.1in}
        \epsfxsize=2.8in
    \hspace{-0.0in}    \epsfbox{eCoeExoDmd.eps}
    \end{minipage}
    \begin{minipage}{2.8in}
        \epsfxsize=2.8in
    \hspace{-0.0in}    \epsfbox{eCoeEndDmd.eps}
    \end{minipage}
\end{minipage}
\vspace{.05in} \caption{Relative savings as a function of exchange
rate parameter in the Basic Model (right) and in the Price Model
(left).} \label{fig:exch} \vspace{-.2in}
\end{center}
\end{figure}


{\noindent \em Demand coefficient of variation.} The coefficient of demand has little effect on the savings of financial hedging and slightly reduces the savings of operational hedging, as illustrated in Figure \ref{fig:demand}.  The flat savings of financial hedging is the average result of two opposite effects of demand variance, reduced cash flow at low capacity and increased variance at high capacity.  When capacity is low, increasing demand variance mainly reduces the cash flow.  This has the same effect as reducing prices or increasing price sensitivity, resulting in increases of financial hedging benefit, due to convexity of exponential utility function's derivative.  When capacity is high, increasing demand variance mainly increase variance of cash flow. Since this part of cash flow variance is not contributed by exchange rate, thus, cannot be hedged by the financial contracts, it makes financial hedging less beneficial.  Both effects cancel each other on average, resulting in the flat savings of financial hedging.  %The decreasing of operational hedging savings is mainly caused by the increased transportation costs.  Higher demand variance results in more capacity mismatch between the two markets, thus more transshipped goods are needed.
\begin{figure}[ht]
\begin{center}
\begin{minipage}{6in}
    \begin{minipage}{3.1in}
        \epsfxsize=2.8in
    \hspace{-0.0in}    \epsfbox{dCoeExoDmd.eps}
    \end{minipage}
    \begin{minipage}{2.8in}
        \epsfxsize=2.8in
    \hspace{-0.0in}    \epsfbox{dCoeEndDmd.eps}
    \end{minipage}
\end{minipage}
\vspace{.05in} \caption{Relative savings as a function of demand coefficient of variation in Basic Model (left) and in Price Model (right).} \label{fig:demand} \vspace{-.2in}
\end{center}
\end{figure}

\noindent {\em Horizon of financial hedging}
The longer number of periods of financial hedging, the bigger the benefits.  Obviously, it becomes increasingly difficult and increasingly costly (due to additional reserves required) to hedge very far into the future.  Figure \ref{fig:HorizonAndStructure} left shows the relative benefits of financial hedging for different number of periods.


\noindent {\em The structure of supply chain}  Most of the results we presented here are for two markets (domestic and overseas).  Only one overseas market increases the values of both operational and financial hedging.  Figure \ref{fig:HorizonAndStructure} right shows the results as a function of demand coefficient of variability.
\begin{figure}[ht]
\begin{center}
\begin{minipage}{6in}
    \begin{minipage}{3.1in}
        \epsfxsize=2.8in
    \hspace{-0.0in}    \epsfbox{hedgingPeriods.eps}
    \end{minipage}
    \begin{minipage}{2.8in}
        \epsfxsize=2.8in
    \hspace{-0.0in}    \epsfbox{foreignDmdonly.eps}
    \end{minipage}
\end{minipage}
\vspace{.05in} \caption{Financial hedging horizon effect(left) and relative savings with overseas market demand only (right).} \label{fig:HorizonAndStructure} \vspace{-.2in}
\end{center}
\end{figure}

\OUT{
\subsection{Dynamic Behavior by Centralized Study}
The centralized study reveals interesting dynamic behavior of savings realized via financial and operational hedging, and also of the optimal capacity allocation.

{\noindent \em Initial exchange rate and planning horizon.} Even though the benefits of financial and operational hedging are very depending on the initial exchange rate, they seem to converge to a common value, as shown in Figures \ref{fig:startExRateFi} and
 \ref{fig:startExRateOp}.  This may be attributed to the convergence of exchange-rate distribution when time horizon increases. Therefore, the longer the planning horizon, the more the stationary distribution of the exchange rate dominates the savings of the hedging.  However, given a short planning horizon, financial hedging and operational hedging have different dynamics.



Consider financial hedging for short horizons , as shown in Figure \ref{fig:startExRateFi}.  The median starting exchange rate results in higher savings than extreme starting rates do since the short-horizon variance of exchange rates is lower at extreme values, implied by the assumption that the exchange rate is mean-reverse (thus, at very low or very high level, the exchange rate has not much room to change in the short term).
\begin{figure}[ht]
\begin{center}
\begin{minipage}{6in}
    \begin{minipage}{3.1in}
        \epsfxsize=2.8in
    \hspace{-0.5in}    \epsfbox{startExRateFi.eps}
    \end{minipage}
    \begin{minipage}{2.8in}
        \epsfxsize=2.8in
    \hspace{-0.5in}    \epsfbox{startExRateFiPrice.eps}
    \end{minipage}
\end{minipage}
\vspace{.05in} \caption{Financial hedging savings as a function of planning horizon in the Basic Model (left) and in the Price Model (right).} \label{fig:startExRateFi} \vspace{-.2in}
\end{center}
\end{figure}


In short horizon operational hedging, as shown in Figure \ref{fig:startExRateOp}, the low starting exchange rate results in higher savings because, in anticipation of an imminent rise in exchange rates, leading to a high overseas profit margin, operational hedging allocates higher capacity overseas to take advantage of this situation; a high starting exchange rate has the opposite effect (but for the same reasons).
\begin{figure}[ht]
\begin{center}
\begin{minipage}{6in}
    \begin{minipage}{3.1in}
        \epsfxsize=2.8in
    \hspace{-0.5in}    \epsfbox{startExRateOp.eps}
    \end{minipage}
    \begin{minipage}{2.8in}
        \epsfxsize=2.8in
    \hspace{-0.5in}    \epsfbox{startExRateOpPrice.eps}
    \end{minipage}
\end{minipage}
\vspace{.05in} \caption{Operational hedging savings as a function of planning horizon in the Basic Model (left) and in the Price Model (right).} \label{fig:startExRateOp} \vspace{-.2in}
\end{center}
\end{figure}


{\noindent \em Optimal domestic capacity allocation of operational hedging.} For a long planing horizon, the domestic capacity allocation converges to a single value, as shown in Figure \ref{fig:StExCap}, due to the convergence of exchange rate distribution.  However, the starting exchange rate has a big impact on the allocation of capacity for short planning horizons, as illustrated in Figure \ref{fig:StExCap}.  In particular, when exchange rates are near the upper bound, optimal domestic capacity is higher because exchange rates are expected to go down, overseas profit margins are anticipated to drop, and thus less overseas capacity is allocated.  In contrast, when exchange near the lower bound, the optimal domestic capacity is lower, because exchange rates are expected to rise, causing overseas profit margins to increase.
\begin{figure}[ht]
\begin{center}
\begin{minipage}{6in}
    \begin{minipage}{3.1in}
        \epsfxsize=2.8in
    \hspace{-0.5in}    \epsfbox{domesticCap.eps}
    \end{minipage}
    \begin{minipage}{2.8in}
        \epsfxsize=2.8in
    \hspace{-0.5in}    \epsfbox{domesticCapPrice.eps}
    \end{minipage}
\end{minipage}
\vspace{.05in} \caption{Domestic capacity in the operational hedging as a function of planning horizon in the Basic Model (left) and in the Price Model (right).}\label{fig:StExCap}
\vspace{-.2in}
\end{center}
\end{figure}


The value of converging domestic capacity depends on the variance of exchange rate.  In most case when transshipment cost is moderate, higher variance of exchange rates results in lower domestic capacity, with corresponding higher overseas capacity, because more overseas production is needed to reduce the variance of cash
 flow when exchange rate variance is high, as illustrated in Figure
\ref{fig:ExVCap}.
\begin{figure}[ht]
\begin{center}
\begin{minipage}{6in}
    \begin{minipage}{3.1in}
        \epsfxsize=2.8in
    \hspace{-0.5in}    \epsfbox{domesticCapExV.eps}
    \end{minipage}
    \begin{minipage}{2.8in}
        \epsfxsize=2.8in
    \hspace{-0.5in}    \epsfbox{domesticCapExVPrice.eps}
    \end{minipage}
\end{minipage}
\vspace{.05in} \caption{Domestic capacity in operational hedging as a function of planning horizon in the Basic Model (left) and in the Price Model (right).}\label{fig:ExVCap} \vspace{-.2in}
\end{center}
\end{figure}
}

\OUT{
\section{Comparison and Discussion }
Compare to Porteus and Chowdary. }





\section{Conclusion \label{sect:conclusion}}

The paper seeks to analyze,the relative strengths and weaknesses of financial hedging and operational hedging.  We describe dynamic finite capacity models that allow to capture the relative benefits of both types of hedgings separately and jointly.  The structural properties of the models are derived.  They allow to describe the intuitive behavior of the model with respect to operational policy and somewhat less intuitive properties of financial hedging.  The structural properties also allow us to evaluate various contributing factors numerically.  The structural properties describe sufficient conditions for submodularity, the special structures of financial hedging contract, and analytical optimal solutions to the production and transshipment problem.  We show that with exponential utility, financial hedging can bring the utility of expected value of various scenarios up to utility of expected value (which is the highest one could expect).  For the special case birth-death process of exchange rates, we show that forward contracts are optimal.  The optimal financial hedging unexpectedly changes the relationship between domestic and overseas capacities in some rare situations.  Intuitively, domestic and overseas capacities are substitutable and we provide a sufficient condition for their submodularity.  But in rare cases, the two can become supplementary through optimal financial hedging, as shown in our example.
In general, our results hold for both the case when price is exogenous and also when it is endogenously decided.

Some specific conclusions follow from our numerical study.
For supply chains with multiple locations of demand (including the place when production takes place), operational hedging tends to dominate financial hedging, in terms of savings, except when the exchange rate variance is very high and transshipment cost is very low.  Operational hedging is especially beneficial when capacity is high, risk-averseness is moderate, price not too high (or alternatively price sensitivity high), and exchange rate variance high.  With the inclusion of transportation cost, operational hedging increases MRN's total utility by not only reducing the variability of cash flow, but also by increasing profitability.  Furthermore, the marginal benefit of financial hedging is quite small once operational hedging has used.  Financial hedging is especially beneficial with high variance of exchange rates, but also with low sales price, and moderate risk aversion.  Maybe somewhat surprisingly, operational hedging is a significant tool for the MRN to reduce the exchange rate risk and maximizing her total utility.  This result does not support the seemingly logical conclusion that low cost financial hedging may be sufficient.  (Multiple producers increasingly use operational hedging, which is especially visible in case of automakers).  We identify specific situations that are most appropriate for use of financial hedging and for use of operational hedging.  Two factors are not taken into account across most of the examples.  First, transportation costs very strongly favor operational hedging and we omitted them except for one example that illustrates the effect of transportation costs.  Second, operational hedging requires additional initial investment, as building two plants is usually more expensive than a single one.




\section{Appendix}

\noindent {\bf Proof of Lemma \ref{lem:pcev-bond}: }
We prove the $\V_n(\cdot)$'s independence on $\beta_n$ by induction, the other results will follow from the induction proof process. We first apply variable substitution $\beta_i^\prime = \beta_i + w_i^\prime$ for all $i\in \{0,\cdots,n\}$. When $n=1$, we have
 \begin{eqnarray*}
 \begin{array}{lll}
 \mathcal{U}_1(f_1,f_0|\beta_1) &= &\max_{\beta_0^\prime} \{\E [ -(1+r_f)\rho_1 \exp(- \frac{f_1 + \beta_1^\prime - \beta_0^\prime/(1+r_f)}{\rho_1}) - \rho_0 \exp(-\frac{f_0+ \beta_0^\prime}{\rho_0})]\}\\
&=& \max_{\beta_0^\prime} \{\E [ -(1+r_f)\rho_1 \exp(- \frac{f_1 + \beta_1^\prime - \beta_0^\prime/(1+r_f)}{\rho_1}) - \rho_0 \exp(-\frac{\V_0(f_0)+ \beta_0^\prime}{\rho_0})]\},
 \end{array}
\end{eqnarray*}
where the second equation follows from (\ref{eqn:PCEV}) and (\ref{eqn:pcev-Xn-gen}).  The first order optimality condition on $\beta_0^\prime$ is
\[\frac{\V_0(f_0)+ \beta_0^\prime}{\rho_0}=\frac{f_1 + \beta_1^\prime - \beta_0^\prime/(1+r_f)}{\rho_1} .\]
Substituting back the original decisions, we get optimal investment decision (\ref{eqn:beta-gen}) for $n=1$, the corresponding maximum utility is then (\ref{eqn:Un-gen}). The PCEV in (\ref{eqn:pcev-gen}) follows from (\ref{eqn:PCEV}) and $\V_1(f_1,f_0|\beta_1$ is independent of $\beta_1$. This concludes the proof of induction  base, i.e., the truth for $n=1$.

Now suppose $\V_{n-1}(f_{n-1},\cdots, f_0|\beta_{n-1})$ is independent of $\beta_{n-1}$ for any $n-1 \geq 0$, we proceed to prove the $\V_n(\cdot)$'s independence on $\beta_n$ and all the other results.  By (\ref{eqn:bondInv-rho}) and variable subsitution, the left side of (\ref{eqn:Un-gen}) becomes
\[
\mathcal{U}_n^\prime (f_n, \cdots, f_0|\beta_n^\prime) = \max_{\beta_{i}^\prime \forall 0 \leq i < n} \{\E [ \sum_{i=0}^n \big{(}-(1+r_f)^i\rho_i \exp (-\frac{f_i + \beta_{i}^\prime -\beta^\prime_{i-1}/(1+r_f)}{\rho_i})\big{)}]\}\]
By (\ref{eqn:bondInv-rho}) for $n-1$, the objective function of the above maximization problem becomes
\begin{eqnarray*}
\E [ -(1+r_f)^n\rho_n \exp (-\frac{f_i + \beta_{i}^\prime -\beta^\prime_{n-1}/(1+r_f)}{\rho_n})+\mathcal{U}_{n-1}^\prime(f_{n-1},\cdots, f_0 |\beta_{n-1}^\prime)],
\end{eqnarray*}
where the only decision is $\beta_{n-1}^\prime$.  By the induction assumption, (\ref{eqn:PCEV}), and (\ref{eqn:Un-Xn}) in Lemma~\ref{lem:pcev-Xn-gen}, for $n-1$, the function inside the expectation becomes
\begin{eqnarray*}
-(1+r_f)^n\rho_n \exp (-\frac{f_i + \beta_{i}^\prime -\beta^\prime_{n-1}/(1+r_f)}{\rho_n})-R_{n-1} (1+r_f)^{n-1} \exp (-  \frac{ \beta_{n-1}^\prime + \V_{n-1}(f_{n-1},\cdots, f_0)}{R_{n-1}}).
\end{eqnarray*}
The first order optimality condition implies
\[\frac{\beta_{n-1}^\prime + \V_{n-1}(f_{n-1},\cdots, f_0)}{R_{n-1}}= \frac{f_n + \beta_{n}^\prime -\beta^\prime_{n-1}/(1+r_f)}{\rho_n} . \]
 Substituting the original decision back results (\ref{eqn:beta-gen}) for $n$. This equation implies that the adjusted cash consumptions are equalized between two periods.  Solving this equation results the optimal bond investment
 \[\beta_{n-1}^\prime= (f_n + \beta_n^\prime - \V_{n-1}(f_{n-1},\cdots,f_0)\rho_n/R_{n-1})\frac{R_{n-1}}{R_n}\]

 This optimal decision implies that the equal adjusted cash consumption in period $n$ or $n-1$ is $ (f_n + \beta_n^\prime - \V_{n-1}(f_{n-1},\cdots,f_0)\rho_n/R_{n-1})/R_n+ \V_{n-1}(f_{n-1},\cdots, f_0)/R_{n-1}$. By simplification, it becomes
 \[
(f_n + \beta_n^\prime  + \V_{n-1}(f_{n-1},\cdots,f_0) \frac{(R_n-\rho_n)}{R_{n-1}})/R_n  = \frac{f_n + \beta_n^\prime - \V_{n-1}(f_{n-1},\cdots,f_0)/(1+r_f)}{R_n},
 \]
where the last equation follows from $(R_n-\rho_n)/R_{n-1} = 1/(1+r_f)$. Substituting this optimal consumption back to the objective function and then the original decision back, we have (\ref{eqn:Un-gen}) for $n$.
Finally, using (\ref{eqn:PCEV}) we obtain (\ref{eqn:pcev-gen}) for $n$, which implies that $\V_n(\cdot)$ is independence of $\beta_n$.
\qed

\medskip
\noindent {\bf Proof of Lemma \ref{lem:pcev-FH}: }
We prove the PCEV's independence on $\beta_n(s_n)$ by induction, the other results will follow from the induction proof process. We first apply variable substitution $\beta_i^\prime(S_i) = \beta_i(S_i) + w_i^\prime$ for all $i\in \{0,\cdots,n\}$. When $n=1$ and by (\ref{eqn:maxU-FH}), the left side of (\ref{eqn:Un-FH}) becomes
\[
 \max_{\beta_0^\prime(S_0)} \{ -(1+r_f)\rho_1 \exp(- \frac{f_1 + \beta_1^\prime(s_1) - \frac{\E[\beta_0^\prime(S_0)|S_1=s_1]}{1+r_f}}{\rho_1}) - \E [\rho_0 \exp(-\frac{f_0+ \beta_0^\prime(S_0)}{\rho_0})|S_1=s_1]\}. \]
 By (\ref{eqn:PCEV}) and (\ref{eqn:pcev-Xn-gen}), the objective function of the above maximization problem becomes
\[-(1+r_f)\rho_1 \exp(- \frac{f_1 + \beta_1^\prime(s_1) - \frac{\E[\beta_0^\prime(S_0)|S_1=s_1]}{1+r_f}}{\rho_1}) - \E[\rho_0 \exp(-\frac{\V_0(f_0|S_0)+ \beta_0^\prime(S_0)}{\rho_0})|S_1=s_1]\}.
 \]
By point-wise maximization,  the first order optimality condition on $\beta_0^\prime(s_0)$ is
\[\frac{f_1 + \beta_1^\prime(s_1) - \E[\beta_0^\prime(S_0)|S_1=s_1]/(1+r_f)}{\rho_1} =\frac{\V_0(f_0|S_0=s_0)+ \beta_0^\prime(s_0)}{\rho_0}.\]
This equation means that the optimal derivative investment equalizes the adjusted consumption cash not only between two periods, but also among all possible realizations of the exchange rate (the right side of the equation).  Substituting the original decision back to the above equation results (\ref{eqn:beta-FH}) for $n=1$. Taking expectation of both sides conditioning on $S_1=s_1$ results
\[\frac{f_1 + \beta_1^\prime(s_1) - \E[\beta_0^\prime(S_0)|S_1=s_1]/(1+r_f)}{\rho_1} =\frac{\E[\V_0(f_0)|S_1=s_1]+ \E[\beta_0^\prime(S_0)|S_1=s_1]}{\rho_0}.\]
Simplifying this equations to obtain the price of the optimal derivative
\[ \frac{\E[\beta_0^\prime(S_0)|S_1=s_1]}{1+r_f} = (f_1 + \beta_1^\prime(s_1) -\E[\V(f_0)|S_1=s_1]\frac{\rho_1}{\rho_0}) \frac{\rho_0}{(1+r_f)R_1}.\]
We substitute the price back in the left side of the first order optimality condition, and obtain, for both periods and all scenarios of exchange rate realization, the optimal adjusted cash consumption as follow
\[\frac{f_1 + \beta_1^\prime(s_1)}{\rho_1} (1-\frac{\rho_0}{(1+r_f)R_1}) +\frac{\E[\V_0(f_0)|S_1=s_1]}{(1+r_f)R_1} = \frac{f_1 + \beta_1^\prime(s_1)+\E[\V_0(f_0)|S_1=s_1]/(1+r_f) }{R_1},\]
where the last equation follows from $1-\rho_0/[(1+r_f)R_1] = \rho_1/R_1$. We substitute this consumption cash into the objective function and the original decision variable back, thus obtain (\ref{eqn:Un-FH}) for $n=1$. This and (\ref{eqn:PCEV}) imply (\ref{eqn:pcev-FH}) for $n=1$, and consequently the PCEV is independent of $\beta_1(s_1)$. This concludes the proof of induction  base, i.e., the truth for $n=1$.

Now suppose $\V_{n-1}(f_{n-1},\cdots, f_0|\beta_{n-1}(s_{n-1}))$ is independent of $\beta_{n-1}(s_{n-1})$ for any $n-1 \geq 0$, we proceed to prove that $\V_n(f_n, \cdots, f_0|\beta_{n}(s_n))$ is independence of $\beta_n(s_n)$ and all the other results.  By (\ref{eqn:maxU-FH}) and variable subsitution, the left side of (\ref{eqn:Un-FH}) becomes
\[
\mathcal{U}_n^\prime (f_n, \cdots, f_0|\beta_n^\prime(s_n)) = \max_{\beta_{i}^\prime \forall 0 \leq i < n} \{\E [ \sum_{i=0}^n \big{(}-(1+r_f)^i\rho_i \exp (-\frac{f_i + \beta_{i}^\prime -\frac{\E[\beta^\prime_{i-1}(S_{i-1})|S_i]}{(1+r_f)}}{\rho_i})\big{)}|S_n=s_n]\}.\]
By (\ref{eqn:maxU-FH}) for $n-1$, the objective function of the above maximization problem becomes
\begin{eqnarray*}
 -(1+r_f)^n\rho_n \exp (-\frac{f_i + \beta_{i}^\prime(s_n) -\E[\beta^\prime_{n-1}(S_{n-1})|s_n]/(1+r_f)}{\rho_n})+\E[\mathcal{U}_{n-1}^\prime(f_{n-1},\cdots, f_0 |\beta_{n-1}^\prime(S_{n-1}))|s_n],
\end{eqnarray*}
where the only decision is derivative payoff function $\beta_{n-1}^\prime(S_{n-1})$.  By the induction assumption, (\ref{eqn:PCEV}), and (\ref{eqn:Un-Xn}) for $n-1$, the objective function becomes
\begin{align}
&-(1+r_f)^n\rho_n \exp (-\frac{f_i + \beta_{i}^\prime -\E[\beta^\prime_{n-1}(S_{n-1})|s_n]/(1+r_f)}{\rho_n})\\
&-R_{n-1} (1+r_f)^{n-1} \E[\exp (-  \frac{ \beta_{n-1}^\prime(S_{n-1}) + \V_{n-1}(f_{n-1},\cdots, f_0|S_{n-1})}{R_{n-1}})|s_n].
\end{align}
The first order optimality condition on $\beta^\prime(s_{n-1})$ is
\[ \frac{f_n + \beta_{n}^\prime(s_n) -\E[\beta^\prime_{n-1}(S_{n-1})|s_n]/(1+r_f)}{\rho_n} =\frac{\beta_{n-1}^\prime(s_{n-1}) + \V_{n-1}(f_{n-1},\cdots, f_0|s_{n-1})}{R_{n-1}}. \]
This equation becomes (\ref{eqn:beta-FH}) for $n$ when the original decisions are substituted back. Taking expectation of both sides conditioning on $s_n$, we have
\[ \frac{f_n + \beta_{n}^\prime(s_n) -\E[\beta^\prime_{n-1}(S_{n-1})|s_n]/(1+r_f)}{\rho_n} =\frac{\E[\beta_{n-1}^\prime(S_{n-1})|s_n] + \E[\V_{n-1}(f_{n-1},\cdots, f_0)|s_n]}{R_{n-1}}. \]
This equation implies that the optimal derivative price is
\[
\frac{\E[\beta_{n-1}^\prime(S_{n-1})|s_n]}{1+r_f} = (f_n + \beta_{n}^\prime(s_n)-\E[\V_{n-1}(f_{n-1},\cdots, f_0)|s_n] \frac{\rho_n}{R_{n-1}})\frac{R_{n-1}}{(1+r_f)R_n}.
\]
This derivative price  implies that the consumption cash of period $n$ is
\[\frac{f_n + \beta_n^\prime(s_n)}{\rho_n} (1-\frac{R_{n-1}}{(1+r_f)R_n}) +\frac{\E[\V_{n-1}(f_{n-1}, \cdots, f_0)|s_n]}{(1+r_f)R_n} = \frac{f_n + \beta_n^\prime(s_n)+\frac{\E[\V_{n-1}(f_{n-1}, \cdots, f_0)|s_n]}{(1+r_f)} }{\rho_n},
\]
where the equation follows from $(1-\frac{R_{n-1}}{(1+r_f)R_n}) = \rho_n/R_n$. This cash consumption in period $n$ is also the one in period $n-1$ at any realization of the exchange rate because of the optimal derivative investment. Substituting these consumption cashes into objective function yields (\ref{eqn:Un-FH}) for $n$.  Finally, using (\ref{eqn:Un-FH}) and (\ref{eqn:PCEV}) we obtain (\ref{eqn:pcev-FH}) for $n$, consequently the PCEV is independent of $\beta_n(s_n)$.
\qed

\medskip

\noindent {\bf Proof of Theorem \ref{cor:fwd}: }
We prove this corollary by showing that the optimal financial contract payoff in $\beta_{n-1}(s_{n-1})$ in (\ref{eqn:beta-FH}) can be obtained by bond and forward investments when the exchange rate in each state goes either up or down into the next period. To simply the notation, we let $p$ ($q$) be the probability that the exchange rate goes to $s_{n-1}^1$ up ($s_{n-1}^2$ down). The corresponding payoff at those states are $\beta^1$ and $\beta^2$. Furthermore, for simplicity, denote:
$a_i = \V_{n-1}(f_{n-1},\cdots,f_0|s_{n-1}^i)+w_{n-1}^{\prime}, i\in \{1,2\}$;
$a_3 = \frac{R_{n-1}}{\rho_n}(f_n+\beta_n(s_n)+w_n^{\prime} $;
$\alpha_3 =  \frac{R_{n-1}}{(1+r_f)\rho_n}$. we expressed (\ref{eqn:beta-FH}) using these notation as follows
\begin{align} \beta^1 + a_1 &= a_3 - \alpha_3 (p\beta^1+q\beta^2),\\
\beta^2 + a_2 &= a_3 - \alpha_3 (p\beta^1+q\beta^2).
\end{align}
The solutions of the above equations are:
\begin{align} \beta^1 &= \frac{a_3 +a_2 \alpha_3q-a_1(1+\alpha_3q)}{1+\alpha_3},\\
\beta^2 &= \frac{a_3 +a_1 \alpha_3p-a_2(1+\alpha_3p)}{1+\alpha_3}.
\end{align}
These payoffs can be constructed by entering a forward contract with  $\frac{\beta^1-\beta^2}{s_{n-1}^1-s_{n-1}^2}$ shares and buying $\frac{\beta^1-s_{n-1}^1(\beta^1-\beta^2)/(s_{n-1}^1-s_{n-1}^2)}{1+r_f}$ risk-free bonds. \qed
\medskip

\noindent {\bf Proof of Lemma \ref{lemma:conProfit}: } First, $f$ is non-decreasing in $(\bold{k},\xiv)$ because
$A(\bold{k}_1, \xiv_1) \subseteq A(\bold{k}_2, \xiv_2)$ when $(\bold k_1, \xiv_1) \leq (\bold{k}_2, \xiv_2)$.
Second, we prove the concavity of $f$.  Assume $f(s,\bold {k}_i, \bold{\xiv}_i) = J(\bold {z}_i, \bold{x}_i,s)$, where $(\bold{z}_i, \bold{x}_i)\in A(\bold{k}_i, \xiv_i)$.  Since $((1-\lambda)\bold {z}_1 + \lambda \bold {z}_2, (1-\lambda)\bold {x}_1 + \lambda \bold {x}_2) \in A((1-\lambda)\bold{k}_1 + \lambda \bold {k}_2, (1-\lambda)\xiv_1 + \lambda \xiv_2)$, it must follow that $f(s,(1-\lambda)\bold{k}_1 + \lambda \bold {k}_2, (1-\lambda)\xiv_1 + \lambda \xiv_2)) \geq J((1-\lambda)\bold {z}_1 + \lambda \bold {z}_2, (1-\lambda)\bold {x}_1 + \lambda \bold {x}_2,s) \geq (1-\lambda) J(\bold {z}_1, \bold{x}_1,s) + \lambda J(\bold {z}_2, \bold{x}_2,s) = (1-\lambda) f(s,\bold {k}_1, \bold{\xiv}_1) +  \lambda f(s,\bold {k}_2, \bold{\xiv}_2)$.  The last inequality follows because the objective function in (\ref{eqn:profit}) is concave in production decisions $(\bold {x,z})$.\qed

\medskip

\noindent {\bf Proof of Lemma \ref{lemma:subProfit}: }
 For simplicity of the presentation, let $\alpha_{jl}$ be the coefficient of decision variable of objective function in (\ref{eqn:profit}), {\it i.e.}, $\alpha_{dd} = p-c, \; \alpha_{od} = p-cs-t,\; \alpha_{do} = sp -c -t, \; \alpha_{oo} = s(p-c)$.  To prove submodularity, we first write the dual of the production problem:
\begin{eqnarray*}
\begin{array}{lll}
    f(s,\bold k, \xiv) &=& \min_{(\etav, \lambdav) \in B (\alphav)} G (\etav, \lambdav, \bold k ,\xiv)  \\
    G(\etav,\lambdav,\bold k, \xiv) &=& \etav \bold k + \lambdav \xiv \\
    B (\alphav) &=& \{\eta_j+\lambda_l  \leq \alpha_{jl} \; \forall j \in \{d,o\}, \; \forall l \in \{d,o\},\;  \etav \geq 0,\; \lambdav \geq 0 \}
\end{array}
\end{eqnarray*}
To see the relationship between $\bold k$ and $-\xiv$, we substitute $\bar{\xiv} = - \xiv$ and $\bar{\etav} = - \etav$.  Then, the dual problem becomes:
\begin{eqnarray*}
\begin{array}{lll}
    f(s,\bold k, \xiv) &=& \min_{(\bar{\etav}, \lambdav) \in B_1 (\alphav)} G (\bar{\etav}, \lambdav, \bold k ,\bar{\xiv})  \\
    G(\bar{\etav},\lambdav,\bold k, \bar{\xiv}) &=& -\bar{\etav} \bold k - \lambdav \bar{\xiv} \\
     B_1(\alphav) &=& \{-\bar{\eta}_j+\lambda_l  \leq \alpha_{jl} \; \forall j \in \{d,o\}, \; \forall l \in \{d,o\},\;  \bar{\etav} \leq 0,\; \lambdav \geq 0 \}
\end{array}
\end{eqnarray*}
The objective function $G$ is submodular in $(\bar{\etav},\bold k,\lambdav, \bar{\xiv})$ and the constraint set $B_1$  is a sublattice of $(\bar{\etav},\lambdav)$.  Since minimizing submodular function over a sublattice preserves submodularity, $f$ is submodular in $(\bold{k}, -\xiv)$.  \qed

\medskip

\noindent {\bf Proof of Theorem \ref{pro:bSol}: }
Obviously, there exist optimal ($x^d,x^o$) such that $x^dx^o =0$.  Let us first consider scenario $x^d \geq 0, \;x^o =0$.  The problem becomes:
\begin{eqnarray*}
    f(s,\bold {k},\xiv) &=& \bar{p}\xi^d + s\bar{p}\xi^o +\max_{(\xi^d \wedge k^o) \geq x^d \geq 0} \bar{t}^d x^d - \bar{p}(\bar{k}^d + x^d)^- - s\bar{p}(x^d-\bar{k}^o)^+
\end{eqnarray*}
This formulation can be interpreted as follows.  The total profit consists of $\bar{p}\xi^d + s\bar{p}\xi^o$, the profit if all demand were satisfied, and the objective function under maximization, which is the impact of transshipment $x^d$.  $\bar{t}^d x^d$ is the relative cost saving of cross sales.  $\bar{p}(\bar{k}^d + x^d)^-$ is the cost of not being able to meet domestic demand, if $x^d$ is too small.  $s\bar{p}(x^d-\bar{k}^o)^+$ is the cost of not meeting overseas demand, if $x^d$ is too big.

Clearly the objective function is concave in $x^d$, as its first-order derivative decreases in $x^d$ is
%\begin{equation}
    $\bar{t}^d +\bar{p}1_{x^d \leq -\bar{k}^d}-s\bar{p}1_{x^d \geq \bar{k}^o}.$
%\end{equation}

Depending on the relative values of $-\bar{k}^d$ and $\bar{k}^o$, the above function takes one of two forms shown in Figure \ref{figure:xdDeriveBasicModel}.  Marginal profit depends on the sign of each of the three segments of the derivative, which results in the target level of $\bar{x}^d$.
\begin{figure}[ht]
\begin{center}
    \epsfig{file=xdDeriveBasicModel.eps,width=4.5in,height=3.0in}
\end{center}
    \caption{The derivative of objective function on $x^d$}
    \label{figure:xdDeriveBasicModel}
\end{figure}

When $x^d = 0$ and $x^o \geq 0$,  the objective function becomes $\bar{t}^o x^o - s\bar{p}(\bar{k}^o + x^o)^- - \bar{p}(x^o-\bar{k}^d)^+$.  Its derivative is illustrated in Figure \ref{figure:xoDeriveBasicModel}.
\begin{figure}[ht]
\begin{center}
    \epsfig{file=xoDeriveBasicModel.eps,width=4.5in,height=3.0in}
\end{center}
    \caption{The derivative of objective function on $x^o$}
    \label{figure:xoDeriveBasicModel}
\end{figure}
The same analysis can be applied to this scenario, which leads to $\bar{x}^o$.  \qed

\medskip

\noindent {\bf Proof of Lemma \ref{lem:pcev-Xn-gen}: }
We prove (\ref{eqn:beta-Xn}) and (\ref{eqn:Un-Xn}) first, then (\ref{eqn:pcev-Xn-gen}) will follow. By (\ref{eqn:bondInv-rho}), the left side of (\ref{eqn:Un-Xn}) is
\begin{eqnarray*}
\max_{\beta_i, \forall 0 \leq i < n}\{\E[ -w_n \exp( - \frac{f_n + \beta_n - \beta_{n-1}/(1+r_f)}{\rho_n}) - \sum_{i=0}^{n-1}w_i \exp (-\frac{\beta_{i-1} -\beta_{i-1}/(1+r_f)}{\rho_i}) ]\}.
\end{eqnarray*}

To this maximization problem, we apply variable substitutions as follow:
\begin{eqnarray*}
\beta_i^{\prime} = \left\{ \begin{array}{ll}
f_n + \beta_n + w_n^\prime, & \mbox{if}\; i = n; \\
\beta_i + w_i^\prime, & \mbox{if} \;  0\leq i \leq n-1.
\end{array}
\right.
\end{eqnarray*}
This variable substitution transforms the maximization problem to be
\begin{eqnarray*}
\max_{\beta_i^{\prime}, \forall 0 \leq i < n}\{\E[ - \sum_{i=0}^{n}(1+r_f)^i\rho_i \exp (-\frac{\beta^\prime_{i} -\beta^\prime_{i-1}/(1+r_f)}{\rho_i}) ]\}.
\end{eqnarray*}
The first order necessary conditions are $[
\beta^\prime_1 - \beta^\prime_0/(1+r_f)]/\rho_1 = \beta^\prime_0/\rho_0$ for $i=0$, and for $0 < i \leq n-1$, $[\beta^\prime_{i+1} -\beta^\prime_{i}/(1+r_f)]/\rho_{i+1}  = [\beta^\prime_{i} -\beta^\prime_{i-1}/(1+r_f)]/\rho_i$. These equations imply that the optimal decisions satisfy $\beta^\prime_i/R_i = \beta_{i+1}^\prime/R_{i+1}$ for all $ 0\leq i\leq n-1$. As a result, the risk adjusted consumption cash in all periods are equalized at
\[ \frac{\beta_n^\prime-\beta_{n-1}^\prime/(1+r_f)}{\rho_n} = \frac{\beta_n^\prime-\beta_{n}^\prime R_{n-1}/[R_n(1+r_f)]}{\rho_n}  = \frac{\beta_n^\prime}{R_n},\]
 where the last equation follows from $R_{n} = R_{n-1}/(1+r_f) + \rho_n$. As a result, the maximum utility is $\E[-(1+r_f)^nR_n \exp (- \frac{\beta^\prime_n}{R_n} )]$. We obtain (\ref{eqn:beta-Xn}) and (\ref{eqn:Un-Xn}) by substituting back the original decision variables. Finally, the PCEV's independence on $\beta_n$ and  (\ref{eqn:pcev-Xn-gen}) follow from (\ref{eqn:PCEV}).
\qed

\medskip

\noindent {\bf Proof of Lemma \ref{lemma:priceFsub}: }
$f$'s submodularity in $(\bold k, -\xiv)$ is equivalent to $f$'s being supermodular in the pairs $(k^d, -k^o)$, $(k^d, \bar{y}^d)$, $(k^d, \bar{y}^o)$, $(k^o, \bar{y}^d)$, $(k^o,\bar{y}^o)$, and $(\bar{y}^d, -\bar{y}^o)$.  Since supermodularity is preserved under summation and maximization, it suffices to prove that $f^d$ and $f^o$ are supermodular in those pairs.  Without loss of generality, let us consider $f^d$.  Recall that maximum operator on a sublattice preserves submodularity.

First, $f^d$ is supermodular in $(-k^d,k^o)$, $(k^o,\bar{y}^d)$ and $(\bar{y}^d,-\bar{y}^o)$ because $G^d$ is supermodular, and the constraint is a sublattice in $(-k^d,k^o,x^d)$, $(k^o,\bar{y}^d,x^d)$ and $(\bar{y}^d,-\bar{y}^o,x^d)$.

Second, $f^d$'s supermodularity in $(k^d,\bar{y}^d)$ and $(k^d,\bar{y}^o)$ is proved by substituting $x^{dd} = \bar{y}^d - x^d$.  $G^d$ and the constraint set become:
\begin{eqnarray*}
G^d(x^{dd}) &=& \bar{t}^d (\bar{y}^d-x^{dd})- ((k^d-x^{dd}
)^-)^2/2 - ((x^{dd})^-)^2/2 - s((\bar{y}^d
-x^{dd}-k^o+\bar{y}^o)^+)^2/2 \\
&& \bar{y}^d \geq x^{dd} \geq \max(0, \bar{y}^d - k^o).
\end{eqnarray*}
Clearly $G^d$ is supermodular and the constraint is a sublattice in $(x^{dd}, k^d, \bar{y}^d)$ and in $(x^{dd}, k^d, \bar{y}^o)$.

Third, to show $f^d$'s supermodularity in $(k^o,\bar{y}^o)$, substitute $x^{do} = k^o - x^d$.  Then, $G^d$ and the constraint become:
\begin{eqnarray*}
G^d(x^{do}) &=& \bar{t}^d (k^o-x^{do})- ((k^o
-x^{do}+k^d-\bar{y}^d )^-)^2/2 - ((k^o-x^{do})^-)^2/2 -
s((\bar{y}^o
-x^{do})^+)^2/2 \\
&& k^o \geq x^{do} \geq \max(0, k^o-\bar{y}^d).
\end{eqnarray*}
Clearly, $G^d$ is supermodular and the constraint is a sublattice in $(k^o, y^o, x^{do})$.  Thus, $f^d$ is supermodular.  By symmetry, $f^o$'s supermodularity is proved by the same logic. \qed

\medskip

\bibliographystyle{nonumber}
\begin{thebibliography}{1}

\bibitem{Agrawal2000}
Agrawal, V., S. Seshadri. 2000.
\newblock Impact of uncertainty and risk aversion on price and order quantity
  in the newsvendor problem.
\newblock {\em Manufacturing \& Service Operations Management}, {\bf 2}(4):410--423.

\bibitem{Fhedge1}
American Risk and Insurance Association.
\newblock {\em Journal of Risk and Insurance}.

\bibitem{Arrow1951}
Arrow, K.~J. 1951.
\newblock Alternative approaches to the theory of choice in risk-taking
  situations.
\newblock {\em Econometrica}, {\bf 19}(4) 404--437.

\bibitem{Aytekin2004}
Aytekin, U., J.~R. Birge. 2004.
\newblock Optimal investment and production across markets with stochastic
  exchange rates.
\newblock {\em Working paper, August 2004}.




\bibitem{Bandaly_supply_2010}
Bandaly

\bibitem{Chowdhry1999}
Bhagwan, C., J. T.~B. Howe. 1999.
\newblock Corporate risk management for multinational corporations: Financial
  and operational hedging policies.
\newblock {\em European Finance Review}, {\bf 2} 229--246.



\bibitem{Bickel2002}
Bickel, J.~E., D.~Fishman, C.~S. Spetzler. 2002
\newblock corporate risk tolerance: Taking the right risks.
\newblock Technical report, Strategic Decisions Group, Menlo Park, CA.

\bibitem{boyabatli2004}
Boyabatli, B., B. Toktay. 2004.
\newblock {Operational Hedging:  A Review with Discussions.}
\newblock {\em Working Paper, 2004.}

\bibitem{Bouakiz1992}
Bouakiz, M., M.~J. Sobel. 1992.
\newblock Inventory control with an exponential utility criterion.
\newblock {\em Operations Research}, {\bf 40}(3) 603--608.

\bibitem{Brealey2003}
Brealey, R.~A., S.~C. Myers. 2003.
\newblock {\em Principles of Corporate Finance}.
\newblock McGraw-Hill/Irwin.

\bibitem{Caldentey}
Caldentey, R., M. Haugh. 2006.
\newblock Optimal control and hedging of operations in the presence of
  financial markets.
\newblock {\em Mathematics of Operations Research}, {\bf 31}(2) 285-304.

\bibitem{Chen2004}
Chen X., M. Sim, D. Simchi-Levi, P. Sun. 2006.
\newblock Risk aversion in inventory management.
\newblock Working paper, August 2006.


\bibitem{cisco}
Cisco System Inc. 2009. Annual Report.
\newblock page~38.



\bibitem{devalkar_integrated_2010}
Devalkar K. S., R. Anupindi, A. Sinha
\newblock Integrated Optimization of Procurement, Processing and Trade of Commodities in a Network Environment.
\newblock Working paper, University of Michigan, April 2010


\bibitem{Dasu}
Dasu, S., L. Li. 1997.
\newblock Optimal Operating Policies in the Presence of Exchange Rate Variability.
\newblock {\em Management Science}, {\bf 43}(5) 705-722

\bibitem{Ding2004}
Ding, Q., L. Dong, P. Kouvelis. 2007.
\newblock On the integration of production and financial hedging decisions in
  global markets.
\newblock {\em Operations Research}, {\bf 55}(3) 470-489.

\bibitem{Dong2010}
Dong, L., P. Kouvelis, P. Su. 2010.
\newblock Global Facility Network Design with Transshipment and Responsive Pricing.
\newblock {\em Manufacturing \& Service Operations Management}, {\bf 12}(2) 278-298.


\bibitem{Eeckhoudt1995}
Eeckhoudt, L., C. Gollier, H. Schlesinger. 1995.
\newblock The risk-averse (and produent) newsboy.
\newblock {\em Management Science}, {\bf 41}(5) 786--794.

\bibitem{Fishburn1989}
Fishburn, P.~C. 1989.
\newblock Retrospective on the utility theory of von neumann and morgenstern.
\newblock {\em Journal of Risk and Uncertainty}, {\bf 2}(2) 127--157.

\bibitem{Fishburn1982}
Fishburn, P.~C., A. Rubinstein. 1982.
\newblock Time preference.
\newblock {\em International Economic Review}, {\bf 23}(3) 677--694.

\bibitem{Frederick2002}
Frederick, S., G. Loewenstein, T. O'Donoghue. 2002.
\newblock Time discounting and time preference: A critical review.
\newblock {\em Journal of Economic Literature}, {\bf 40}(2) 351--401.

\bibitem{Guar2004}
Gaur, V., S. Seshadri. 2005.
\newblock Hedging Inventory Risk Through Market Instruments.
\newblock {\em M\&SOM}, {\bf 7}{2} 103-120.

\bibitem{Gerber1998}
Gerber, H.~U., G. Pafumi. 1998.
\newblock Utility functions: From risk theory to finance.
\newblock {\em North American Actuarial Journal}, {\bf 2}(3) 74--100.

\bibitem{Huchzermeier}
Huchzermeier, A., M. A. Cohen. 1996.
\newblock Valuing Operational Flexibility under Exchange Rate Risk
\newblock {\em Operations Research}, {\bf 44}(1) 100-113.

\bibitem{Hu2004a}
Hu, X., I. Duenyas, R. Kapuscinski. 2003.
\newblock Advance demand information and safety capacity as a hedge against
  demand and capacity uncertainty.
\newblock {\em M\&SOM}, {\bf 5}(1) 55-58.

\bibitem{Hu2004b}
Hu, X., I. Duenyas, R. Kapuscinski. 2004.
\newblock Existence of coordinating transshipment prices in a two-location
  inventory model.
\newblock Working paper.

\bibitem{Kazaz2005}
Kazaz, B., M. Dada, H. Herbert. 2005
\newblock Global Production Planning under Exchange-Rate Uncertainty.
\newblock {\em Management Science}, {\bf 57}(7) 1101-1119.



\bibitem{Krishnan1965}
Krishnan, K.~S., V.~R.~K. Rao. 1965.
\newblock Inventory control in n warehouses.
\newblock {\em J. Indust. Eng.}, {\bf 16} 212--215

\bibitem{Levy92}
Levy, H. 1992.
\newblock Stochastic dominance and expected untility: Survey and analysis.
\newblock {\em Management Science}, 38(4):555--593.

\bibitem{Li2001}
Li, L., E. L. Porteus, H. Zhang. 2001.
\newblock Optimal Operating Policies for Multiplant Stochastic Manufacturing
  Systems in a Changing Environment.
\newblock {\em Management Science}, {\bf 47}(11) 1539--1551.

\bibitem{Manganelli2001}
Manganelli, S., R. F. Engle. 2001
\newblock Value at risk models in finance.
\newblock Working Paper No 75, European Centra Bank.

\bibitem{Pollak}
Pollak, R. A. 1973.
\newblock The risk independence axiom.
\newblock {\em Econometrica}, {\bf 41}(1):35--39.

\bibitem{Prakash1976}
Prakash, P. 1977.
\newblock On the consistency of a gambler with time preference.
\newblock {\em Journal of Economic Theory}, {\bf 15}(1) 92--98.

\bibitem{Pratt1964}
Pratt, J.~W. 1964.
\newblock Risk aversion in the small and in the large.
\newblock {\em Econometrica}, {\bf 32}(1) 122--136.

\bibitem{Fhedge2}
{\em Risk Management}. Risk Management Society Publishing, Inc.

\bibitem{Samuelson1937}
Samuelson, P. A. 1937.
\newblock A note on measurement of utility.
\newblock {\em The Review of Economic Studies}, {\bf 4}(2) 155--161.

\bibitem{Samuelson1952}
Samuelson, P.~A. 1952
\newblock Probability, utility, and the independece axiom.
\newblock {\em Econometrica}, {\bf 20}(4) 670--678.

\bibitem{Smith1998}
Smith, J. E. 1988.
\newblock Evaluating income streams: A decision analysis approach.
\newblock {\em Management Science}, {\bf 44}(12) 1690-1706.

\bibitem{Smith2004}
Smith, J.~E. 2004.
\newblock Risk sharing; fiduciary duty, and corporate risk attitudes.
\newblock {\em Decision Analysis}, {\bf 1}(2) 114-127.

\bibitem{Sobel2007}
Sobel, M. J. 2006.
\newblock Discounting and Risk Neutrality.
\newblock Case Western Reserve University, Technical Report, September 2006.

\bibitem{Steinbach2001}
Steinbach, M. C. 2001.
\newblock Markowitz revisited: Mean-variance models in financial portfolio
  analysis.
\newblock {\em SIAM Review}, {\bf 41}(1):31--85.

\bibitem{Tagaras1992}
Tagaras, G. 1989.
\newblock Effects of pooling on the optimization and service levels of
  two-location inventory system.
\newblock {\em IIE. Trans.}, {\bf 21} 250--257.

\bibitem{tapiero_value_2005}
Tapiero C. S. 2005
\newblock Value at risk and inventory control.
\newblock {\em European Journal of Operational Research}, {\bf 163} (3):769--775.


\bibitem{vanmieghem2003}
Van Mieghem, J. 2003.
\newblock {Capacity Management, Investment, and Hedging: A Review and Recent Developments.}
\newblock {\em M\&SOM}, {\bf 5}(4), 269-302.

\end{thebibliography}

%\bibliographystyle{plain}
%\bibliography{mrn_newsboy}

%% Here starts the e-companion (EC)
%%%%%%%%%%%%%%%%%%%%%%%%%%%%%%%%%%%%%%%%%%%%%%%%%%%%%%%%%%
\ECSwitch

\ECDisclaimer
%%%%%%%%%%%%%%%%%%%%%%%%%%%%%%%%%%%%%%%%%%%%%%%%%%%%%%%%%%

%%% Main head for the e-companion
\ECHead{Proofs of Statements}

\OUT{
\newpage

\setcounter{page}{1}


\section*{On-Line Appendix}
}



\noindent {\bf Proof of Theorem \ref{theo:priceoptimal}: }
There exist optimal ($x^d,x^o$) such that $x^dx^o =0$.  Let us first consider scenario $x^d \geq 0, \;x^o =0$.  The transshipment decision problem is represented by (\ref{eqn:fd}) and (\ref{eqn:Gd}).  The optimal $x^d$ minimizes $G^d$.  $G^d$'s derivative 
\begin{equation}
G'^d(x^d) = \bar{t}^d - (x^d+\bar{k}^d )^- -(x^d - \bar{y}^d)^+
-s(x^d-\bar{k}^o)^+
\end{equation}
is a decreasing and piece-wise linear function, see Figure \ref{fig:xdPrice}.  Its shape depends on the relative positions of $-\bar{k}^d$ and $\bar{k}^o$, {\it i.e.}, the sign of $\bar{K}$.  We consider two cases:
\begin{figure} [ht]
\begin{center}
\epsfig{file=xdDerivePriceModel.eps,width=4.5in,height=4.0in}
\end{center}
\caption{Illustration of $G'^d(x^d)$}\label{fig:xdPrice}
\end{figure}

\noindent (a) $\bar{k}^o + \bar{k}^d = \bar{K} \geq 0$.
\begin{equation} G'^d(x^d) = \left\{
  \begin{array} {ll}
  \bar{t}^d -\bar{k}^d -x^d  \quad & \mbox{ if $x^d \leq -\bar{k}^d
  $} \\
\bar{t}^d  \quad  & \mbox{if $-\bar{k}^d \leq x^d \leq
(\bar{y}^d\wedge
\bar{k}^o) $} \\
\bar{t}^d  -(x^d - \bar{y}^d)1_{\bar{k}^o \geq \bar{y}^d}-s(x^d -
\bar{k}^o)1_{\bar{k}^o \leq \bar{y}^d} \quad  & \mbox{if $
(\bar{y}^d\wedge
\bar{k}^o) \leq x^d \leq (\bar{y}^d \vee \bar{k}^o)   $}\\
\bar{t}^d  -(x^d - \bar{y}^d)-s(x^d - \bar{k}^o) \quad  & \mbox{if
$(\bar{y}^d\vee \bar{k}^o) \leq x^d$}
  \end{array} \right.
\end{equation}
which implies:
\begin{equation}
\bar{x}^d = \left\{
  \begin{array} {ll}
  \bar{t}^d -\bar{k}^d \quad & \mbox{ if $\bar{t}^d \leq 0$} \\
\frac{\bar{t}^d  + \bar{y}^d1_{\bar{k}^o \geq
\bar{y}^d}+s\bar{k}^o1_{\bar{k}^o \leq \bar{y}^d}}{1_{\bar{k}^o
\geq \bar{y}^d}+s1_{\bar{k}^o \leq \bar{y}^d}} \quad  & \mbox{if
$0 \leq \bar{t}^d \leq (s+1)(\bar{y}^d\vee \bar{k}^o)-\bar{y}^d -
s\bar{k}^o
$}\\
(\bar{t}^d  + \bar{y}^d+s\bar{k}^o)/(1+s) \quad  & \mbox{if $
(s+1)(\bar{y}^d\vee \bar{k}^o)-\bar{y}^d - s\bar{k}^o \leq
\bar{t}^d $}
  \end{array} \right.  \label{eqn:barXd1}
\end{equation}

\noindent (b) $\bar{k}^o + \bar{k}^d = \bar{K} \leq 0$:
\begin{equation}
G'^d(x^d) = \left\{
  \begin{array} {ll}
  \bar{t}^d -\bar{k}^d -x^d  \quad & \mbox{ if $x^d \leq
  \bar{k}^o
  $} \\
\bar{t}^d-\bar{k}^d -x^d- s(x^d -\bar{k}^o)  \quad  & \mbox{if
$\bar{k}^o \leq x^d \leq -\bar{k}^d$} \\
\bar{t}^d- s(x^d -\bar{k}^o)  \quad  & \mbox{if $ -\bar{k}^d \leq x^d \leq \bar{y}^d$}\\
\bar{t}^d  -(x^d - \bar{y}^d)-s(x^d - \bar{k}^o) \quad  & \mbox{if
$\bar{y}^d \leq x^d$}
  \end{array} \right.
\end{equation}
The corresponding solution to $G'^d(x^d) = 0$ is:
\begin{equation}
\bar{x}^d = \left\{
  \begin{array} {ll}
  \bar{t}^d -\bar{k}^d \quad & \mbox{ if $\bar{t}^d \leq \bar{K}$} \\
(\bar{t}^d  - \bar{k}^d +s\bar{k}^o)/(1+s)  \quad  & \mbox{if $
\bar{K} \leq  \bar{t}^d \leq -s\bar{K}$}\\
(\bar{t}^d  +s\bar{k}^o)/s \quad  & \mbox{if $ -s\bar{K} \leq
\bar{t}^d \leq  s(\bar{y}^d - \bar{k}^o)$} \\
(\bar{t}^d  + \bar{y}^d+s\bar{k}^o)/(1+s) \quad  & \mbox{if $
s(\bar{y}^d - \bar{k}^o) \leq \bar{t}^d $}
  \end{array} \right.  \label{eqn:barXd2}
\end{equation}
$\bar{x}^d$ in (\ref{eqn:barXd}) is obtained by combining (\ref{eqn:barXd1}) and (\ref{eqn:barXd2}).
\\
When $x^d = 0$ and $x^o \geq 0$, the role of $x^d$ and $x^o$ is reversed. \qed





\end{document}


